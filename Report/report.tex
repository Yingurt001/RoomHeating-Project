% !TEX program = xelatex
\documentclass[12pt,a4paper]{article}

% === Chinese support ===
\usepackage{xeCJK}
\setCJKmainfont{STSong}
\setCJKsansfont{STHeiti}
\setCJKmonofont{STFangsong}

% === Page layout ===
\usepackage[top=2.5cm, bottom=2.5cm, left=2.5cm, right=2.5cm]{geometry}

% === Math ===
\usepackage{amsmath, amssymb, amsthm}

% === Graphics ===
\usepackage{graphicx}
\usepackage{float}
\usepackage{subcaption}

% === Tables ===
\usepackage{booktabs}
\usepackage{array}
\usepackage{multirow}

% === Other ===
\usepackage{hyperref}
\usepackage{xcolor}
% enumitem removed - using default lists

\hypersetup{
    colorlinks=true,
    linkcolor=blue!60!black,
    citecolor=blue!60!black,
    urlcolor=blue!70!black
}

% Figure path
\graphicspath{{../Code/results/}}

% === Title ===
\title{\textbf{房间供暖系统的有效控制} \\[0.5em]
\large Effective Control of Room Heating}
\author{MATH3060/1 — 应用数学方向,数学小组项目 \\
University of Nottingham, School of Mathematical Sciences \\
2026 年春季}
\date{}

\begin{document}
\maketitle

\begin{abstract}
本报告研究房间供暖系统的数学建模与最优控制问题。我们建立了三种递进复杂度的物理模型——集总参数ODE模型、一维热方程PDE模型和二维热方程PDE模型——以及一个多房间整栋楼耦合模型,并在每种模型上比较了四种控制策略:Bang-Bang开关控制、PID连续反馈控制、LQR线性二次最优控制和Pontryagin极小值原理。

研究采用统一的评价指标体系(能耗、温度RMSE、超调量、稳定时间、切换次数和统一代价函数),通过大量数值实验得出以下核心结论:
(1)在集总参数模型中,LQR和Pontryagin策略性能接近理论最优,但充分调参的PID可达到相当甚至更低的代价函数值;
(2)当引入空间维度后(PDE模型),恒温器和加热器的\textbf{位置}对控制性能的影响远大于控制策略本身的差异——这是本项目的核心创新发现;
(3)对流效应对PDE模型的可控性起决定性作用,纯扩散模型严重低估实际房间的热响应速度;
(4)在多房间建筑中,外墙房间的自适应增益策略优于统一参数策略,且选择性加热(外墙+中心房间)可实现能耗与舒适度的Pareto最优。

此外,我们严格分析了Bang-Bang控制中的\textbf{Zeno效应}——证明当滞回带$\delta \to 0$时切换次数$N \sim 1/\delta \to \infty$,并给出Filippov滑动模式的正则化;研究了\textbf{不同形状房间}(L形、走廊形)的热控制特性;提出了\textbf{双传感器加权反馈}方案,通过在四分位点放置两个传感器彻底消除了恒温器位置敏感性。模型参数经与CIBSE/ASHRAE实际建筑数据交叉验证,物理合理。

代码完全可复现,托管于 \url{https://github.com/Yingurt001/RoomHeating-Project}。
\end{abstract}

\tableofcontents
\newpage

%% ============================================================
\section{引言}
%% ============================================================

\subsection{问题背景}

房间供暖控制是一个经典的工程问题,也是控制理论的理想教学案例。在英国冬季,室外温度约$T_a = 5\,^\circ\text{C}$,我们需要将室内温度维持在舒适的$T_{\text{set}} = 20\,^\circ\text{C}$。这涉及两个核心问题:

\begin{enumerate}
    \item \textbf{控制策略问题}:采用什么样的加热器控制方式最有效?是简单的开关控制还是连续调节?
    \item \textbf{传感器放置问题}:恒温器应该放在房间的什么位置?距离加热器远近如何影响控制效果?
    \item \textbf{建筑整体策略}:在多房间建筑中,如何协调各房间的供暖以实现全局最优?
\end{enumerate}

\subsection{研究方法}

我们采用从简单到复杂的递进建模方法:

\begin{itemize}
    \item \textbf{ODE模型 (0D)}:将房间视为均匀温度体,用Newton冷却定律建立常微分方程。集中研究控制策略本身的优劣。
    \item \textbf{1D PDE模型}:沿房间长度方向引入空间维度,用一维热方程描述温度分布。研究加热器和恒温器沿一个方向的最优位置。
    \item \textbf{2D PDE模型}:建立矩形房间的二维温度场模型,研究全空间的温度分布和位置优化。
    \item \textbf{多房间建筑模型}:$N$个房间共享内墙的耦合ODE系统,研究建筑级控制策略。
\end{itemize}

在每种模型上,我们比较四种控制策略(表\ref{tab:strategies})。

\begin{table}[H]
\centering
\caption{四种控制策略概述}
\label{tab:strategies}
\begin{tabular}{llll}
\toprule
策略 & 数学核心 & 定位 \\
\midrule
Bang-Bang + 滞回 & 混合动力系统 & 基线对照 \\
PID 控制 & 闭环反馈 $u = K_p e + K_i\!\int\!e\,dt + K_d\dot{e}$ & 工程标准 \\
LQR 最优控制 & Riccati 方程, $J = \int[Qx^2 + Ru^2]dt$ & 数学最优 \\
Pontryagin 极小值 & Hamilton量、伴随方程 & 理论极限 \\
\bottomrule
\end{tabular}
\end{table}

\subsection{评价指标}

所有策略采用统一的六项指标评价(表\ref{tab:metrics})。

\begin{table}[H]
\centering
\caption{统一评价指标体系}
\label{tab:metrics}
\begin{tabular}{lll}
\toprule
指标 & 数学定义 & 物理含义 \\
\midrule
能耗 $E$ & $\int_0^{T_f} u(t)\,dt$ & 总能量消耗 \\
温度 RMSE & $\sqrt{\frac{1}{T_f}\int_0^{T_f}(T-T_{\text{set}})^2\,dt}$ & 舒适度偏差 \\
最大超调 & $\max(T(t) - T_{\text{set}})$ & 峰值偏差 \\
稳定时间 & 首次进入$\pm 0.5\,^\circ\text{C}$并保持的时刻 & 响应速度 \\
切换次数 & 加热器开关转换次数 & 设备磨损\footnotemark \\
统一代价 $J$ & $\int_0^{T_f}[Q(T-T_{\text{set}})^2 + Ru^2]\,dt$ & 综合权衡 \\
\bottomrule
\end{tabular}
\end{table}
\footnotetext{频繁切换不仅加速设备机械磨损,还因启动暂态损耗和浪涌电流导致额外能量消耗\cite{seem1998, ashrae2021}。}

统一代价函数$J$通过权重$Q$和$R$平衡舒适度和能耗,是比较不同策略的最核心指标。

%% ============================================================
\section{ODE 集总参数模型}
%% ============================================================

\subsection{数学建模}

Newton冷却定律将房间视为温度均匀的热容体:
\begin{equation}
\frac{dT}{dt} = -k(T(t) - T_a) + u(t) \label{eq:ode}
\end{equation}
其中$T(t)$为室温,$T_a = 5\,^\circ\text{C}$为室外温度,$k = 0.1\,\text{min}^{-1}$为冷却常数,$u(t) \in [0, U_{\max}]$为加热器控制输入,$U_{\max} = 15\,^\circ\text{C/min}$。

\textbf{稳态分析}:令$dT/dt = 0$,恒定加热$u = U_{\max}$下:
\begin{equation}
T_{\text{ss}} = T_a + \frac{U_{\max}}{k} = 5 + \frac{15}{0.1} = 155\,^\circ\text{C}
\end{equation}
远高于$T_{\text{set}} = 20\,^\circ\text{C}$,表明加热器功率充足。

\subsubsection*{模型合理性验证}

为确保模型结果具有物理可信度,我们将参数与实际建筑数据进行对比\cite{ashrae2021}:

\begin{itemize}
    \item \textbf{热时间常数}:$\tau = 1/k = 10$分钟。根据CIBSE Guide A \cite{cibse2015},轻型建筑(石膏板墙)$\tau \approx 10\text{--}20$分钟,中型建筑(砖墙)$\tau \approx 30\text{--}60$分钟,重型建筑(混凝土)$\tau \approx 60\text{--}120$分钟。我们的参数对应轻型隔热建筑(如活动板房、临时教室),物理合理。
    \item \textbf{加热器功率}:将$U_{\max}$转换为物理功率:$P = U_{\max} \cdot \rho_{\text{air}} c_p V / 60$,其中$V = 5\times4\times2.5 = 50\,\text{m}^3$,$\rho_{\text{air}} = 1.2\,\text{kg/m}^3$,$c_p = 1005\,\text{J/(kg\cdot K)}$,得$P \approx 15.1\,\text{kW}$——对应大型中央供暖系统,与高散热率($k$大)配合合理。
    \item \textbf{稳态工况比}(duty cycle):在稳态下维持$T_{\text{set}}$所需的平均功率为$u_{\text{eq}} = k(T_{\text{set}} - T_a) = 1.5\,^\circ\text{C/min}$,占$U_{\max}$的10\%,即加热器仅需开启10\%的时间——节能的冬季供暖场景,合理。
    \item \textbf{数值精度}:自然冷却($u=0$)时数值解与解析解的最大误差为$2.1 \times 10^{-14}\,^\circ\text{C}$,验证了RK45求解器的精度。
\end{itemize}

\begin{figure}[H]
\centering
\includegraphics[width=\textwidth]{model_plausibility.png}
\caption{模型合理性验证:(a)冷却曲线数值vs解析,(b)不同隔热水平对比,(c)不同室外温度影响,(d)参数与实际数据对照,(e)特征时间尺度,(f)能量平衡验证}
\label{fig:plausibility}
\end{figure}

\textbf{解析解}:无加热($u=0$)时,方程\eqref{eq:ode}的精确解为:
\begin{equation}
T(t) = T_a + (T_0 - T_a)\exp(-kt)
\end{equation}
我们用此解析解验证了数值求解器的精度(最大误差$< 10^{-8}\,^\circ\text{C}$),采用SciPy的\texttt{solve\_ivp}(RK45自适应步长方法)\cite{boyce2021, strogatz2024}。

\subsection{Bang-Bang 开关控制}

最简单的温控方式是带滞回的开关控制\cite{goebel2012, zhang2001}:
\begin{equation}
u(t) = \begin{cases}
U_{\max}, & T(t) < T_{\text{set}} - \delta \\
0, & T(t) > T_{\text{set}} + \delta \\
u(t^-), & \text{其他(保持当前状态)}
\end{cases} \label{eq:bangbang}
\end{equation}

滞回带半宽$\delta$至关重要:$\delta = 0$时系统退化为理想开关,可能出现Zeno现象(有限时间内无限次切换)\cite{zhang2001}。增大$\delta$可减少切换频率但增大温度波动幅度。

\textbf{混合动力系统视角}:方程\eqref{eq:ode}加上切换逻辑\eqref{eq:bangbang}构成\textbf{混合动力系统}(hybrid dynamical system)——连续温度演化与离散开关事件的耦合\cite{goebel2012}。按照Goebel--Sanfelice--Teel框架\cite{goebel2012},可将恒温器形式化为混合自动机:

\begin{equation}
\mathcal{H} = (Q, X, f, \text{Dom}, G, R)
\end{equation}
其中:
\begin{itemize}
    \item 离散状态集 $Q = \{q_{\text{ON}}, q_{\text{OFF}}\}$
    \item 连续状态 $X = \mathbb{R}$(温度$T$)
    \item 流动动力学:$f(q_{\text{ON}}, T) = -k(T - T_a) + U_{\max}$,$f(q_{\text{OFF}}, T) = -k(T - T_a)$
    \item 流动域(flow domain):$\text{Dom}(q_{\text{ON}}) = \{T : T \leq T_{\text{set}} + \delta\}$,$\text{Dom}(q_{\text{OFF}}) = \{T : T \geq T_{\text{set}} - \delta\}$
    \item 守卫条件(guard):$G(q_{\text{ON}} \to q_{\text{OFF}}) = \{T = T_{\text{set}} + \delta\}$,$G(q_{\text{OFF}} \to q_{\text{ON}}) = \{T = T_{\text{set}} - \delta\}$
    \item 重置映射 $R$为恒等(切换时温度不变)
\end{itemize}

\begin{figure}[H]
\centering
\includegraphics[width=0.75\textwidth]{hybrid_automaton.png}
\caption{Bang-Bang恒温器的混合自动机示意图}
\label{fig:hybrid_automaton}
\end{figure}

\subsubsection*{Zeno效应分析}

项目问题文件明确要求:\emph{Zeno效应是否会在模型的某些形式中出现?}这是混合动力系统的核心问题之一\cite{zhang2001, lygeros2003}。

\textbf{解析推导}:对于$\delta > 0$的Bang-Bang控制,加热半周期和冷却半周期分别为:
\begin{align}
t_{\text{heat}} &= \frac{1}{k}\ln\frac{T_{\text{ss}} - T_{\text{low}}}{T_{\text{ss}} - T_{\text{high}}}, \quad T_{\text{ss}} = T_a + \frac{U_{\max}}{k} \\
t_{\text{cool}} &= \frac{1}{k}\ln\frac{T_{\text{high}} - T_a}{T_{\text{low}} - T_a}
\end{align}
其中$T_{\text{low}} = T_{\text{set}} - \delta$,$T_{\text{high}} = T_{\text{set}} + \delta$。

当$\delta \to 0$时,Taylor展开得:
\begin{equation}
t_{\text{heat}} \approx \frac{2\delta}{k(T_{\text{ss}} - T_{\text{set}})}, \quad t_{\text{cool}} \approx \frac{2\delta}{k(T_{\text{set}} - T_a)}
\end{equation}
因此振荡周期$P = t_{\text{heat}} + t_{\text{cool}} = O(\delta)$,切换次数$N \sim T_f / P = O(1/\delta) \to \infty$。

代入参数值:线性系数$C = 2/(k(T_{\text{ss}} - T_{\text{set}})) + 2/(k(T_{\text{set}} - T_a)) \approx 1.48$,即$P \approx 1.48\delta$(分钟)。

\textbf{Zeno分类}:
\begin{itemize}
    \item $\delta > 0$:\textbf{非Zeno}。系统进入周期极限环,周期$P(\delta)$恒定。切换数有限。
    \item $\delta = 0$:\textbf{Zeno点}。当温度首次到达$T = T_{\text{set}}$时($t^* \approx 0.71$分钟),两个模式的向量场方向相反:
    \begin{equation}
    f_{\text{ON}}(T_{\text{set}}) = -k(T_{\text{set}} - T_a) + U_{\max} = +13.5 > 0, \quad f_{\text{OFF}}(T_{\text{set}}) = -k(T_{\text{set}} - T_a) = -1.5 < 0
    \end{equation}
    两个流场都"穿过"切换面,系统进入\textbf{chattering Zeno}——有限时间内无限次切换。
    \item \textbf{Filippov正则化}:在Zeno点,等效控制为$u_{\text{eq}} = k(T_{\text{set}} - T_a) = 1.5\,^\circ\text{C/min}$(仅为$U_{\max}$的10\%),系统进入\textbf{滑动模式}(sliding mode),温度恒定在$T_{\text{set}}$。
\end{itemize}

\textbf{数值验证}:我们对$\delta \in \{0.01, 0.02, 0.05, 0.1, 0.2, 0.5, 1.0, 2.0\}$进行了仿真,表\ref{tab:zeno}展示了解析与数值结果的一致性。

\begin{table}[H]
\centering
\caption{Zeno效应数值验证:解析 vs 数值切换周期}
\label{tab:zeno}
\begin{tabular}{ccccr}
\toprule
$\delta$ ($^\circ$C) & $P_{\text{解析}}$ (min) & $P_{\text{数值}}$ (min) & 误差 & 切换次数 \\
\midrule
2.00 & 2.979 & 2.979 & $<0.01\%$ & 39 \\
1.00 & 1.483 & 1.498 & 1.0\% & 80 \\
0.50 & 0.741 & 0.745 & 0.5\% & 160 \\
0.10 & 0.148 & 0.148 & $<0.01\%$ & 801 \\
0.01 & 0.015 & 0.015 & $<0.01\%$ & 8003 \\
\bottomrule
\end{tabular}
\end{table}

\begin{figure}[H]
\centering
\includegraphics[width=\textwidth]{zeno_analysis.png}
\caption{Zeno效应综合分析:(a)周期$P$与$\delta$的线性关系,(b)切换次数的$1/\delta$发散,(c)加热/冷却半周期,(d)不同$\delta$的温度轨迹,(e)稳态切换间隔,(f)相图与Zeno点分析}
\label{fig:zeno_analysis}
\end{figure}

\begin{figure}[H]
\centering
\includegraphics[width=\textwidth]{zeno_convergence.png}
\caption{(a)解析与数值切换间隔的对比验证,(b)引入切换能耗惩罚$c_s$后的总能耗随$\delta$的变化——$\delta$过小时频繁切换显著增加能耗}
\label{fig:zeno_convergence}
\end{figure}

\textbf{Zeno效应的工程意义}:$\delta = 0$在物理上不可实现(传感器有分辨率限制),但过小的$\delta$会导致切换频率极高,引起继电器疲劳和启动暂态损耗。最优$\delta$应平衡温度波动与切换频率——引入切换能耗惩罚$c_s$后($E = \int u\,dt + c_s \cdot N_{\text{switches}}$),如图\ref{fig:zeno_convergence}(b)所示,$\delta$存在一个能耗最优值,取决于$c_s$的大小。

\textbf{滞回带扫描实验}如表\ref{tab:hysteresis}所示。能耗几乎不随$\delta$变化($\approx 190$),因为系统需要的总热量由散热速率决定。但RMSE随$\delta$增大而显著增加。$\delta = 0.5\,^\circ\text{C}$是合理的工程折衷。

\begin{table}[H]
\centering
\caption{Bang-Bang 滞回带参数扫描}
\label{tab:hysteresis}
\begin{tabular}{cccc}
\toprule
$\delta$ ($^\circ$C) & 切换次数 & RMSE ($^\circ$C) & 能耗 \\
\midrule
0.10 & 1611 & 0.446 & 189.8 \\
0.25 & 644 & 0.465 & 189.9 \\
0.50 & 322 & 0.527 & 189.4 \\
1.00 & 161 & 0.726 & 189.7 \\
2.00 & 80 & 1.235 & 190.4 \\
5.00 & 31 & 2.938 & 183.2 \\
\bottomrule
\end{tabular}
\end{table}

\begin{figure}[H]
\centering
\includegraphics[width=0.85\textwidth]{ode_bangbang.png}
\caption{Bang-Bang控制器响应:温度轨迹与控制信号}
\label{fig:ode_bangbang}
\end{figure}

\subsection{PID 控制}

PID控制器提供连续的控制信号\cite{astrom2021, blasco2012}:
\begin{equation}
u(t) = K_p \cdot e(t) + K_i \int_0^t e(\tau)\,d\tau + K_d \frac{de}{dt}, \quad e(t) = T_{\text{set}} - T(t) \label{eq:pid}
\end{equation}
并约束$u(t) \in [0, U_{\max}]$。当$u$饱和时冻结积分项(anti-windup),防止积分器溢出。

\textbf{调参方法}:对$K_p \in \{0.5, 1, 2, 4, 8\}$,$K_i \in \{0.05, 0.1, 0.3, 0.5, 1.0\}$,$K_d \in \{0, 0.5, 1.0, 2.0\}$共100组参数网格搜索,以统一代价$J$为优化目标。

\textbf{最优参数}:$K_p = 8.0, K_i = 1.0, K_d = 0.0$,对应$J = 25.0$。最优PID的微分项为零(退化为PI控制器),因为本系统是一阶系统,不存在振荡趋势,微分项的预测抑制作用没有必要。

\begin{figure}[H]
\centering
\includegraphics[width=0.85\textwidth]{ode_pid_sweep.png}
\caption{PID参数扫描:不同$K_p$, $K_i$, $K_d$组合的代价函数}
\label{fig:ode_pid_sweep}
\end{figure}

\subsection{LQR 最优控制}

LQR最小化无限时域代价函数\cite{astrom2021, anderson1990}:
\begin{equation}
J = \int_0^\infty \left[ Q(T - T_{\text{set}})^2 + Ru^2 \right] dt \label{eq:lqr_cost}
\end{equation}

引入偏差变量$x = T - T_{\text{set}}$,状态空间为:
\begin{equation}
\dot{x} = Ax + B(u - u_{\text{ss}}), \quad A = -k, \quad B = 1
\end{equation}
其中稳态前馈$u_{\text{ss}} = k(T_{\text{set}} - T_a) = 1.5\,^\circ\text{C/min}$。

代数Riccati方程(ARE):
\begin{equation}
A^T P + PA - PBR^{-1}B^T P + Q = 0 \label{eq:are}
\end{equation}
对标量系统简化为:$P = R\left(-k + \sqrt{k^2 + Q/R}\right)$,最优增益$K = P/R$。

最优控制律为:
\begin{equation}
u^*(t) = u_{\text{ss}} + K \cdot (T_{\text{set}} - T(t)), \quad u^* \in [0, U_{\max}]
\end{equation}

\textbf{Q/R 权重扫描}:扫描$Q \in \{0.01, \ldots, 100\}$和$R \in \{0.001, \ldots, 5\}$共56组参数,生成Pareto前沿(图\ref{fig:ode_pareto})。

\begin{figure}[H]
\centering
\includegraphics[width=0.7\textwidth]{ode_lqr_pareto_sweep.png}
\caption{LQR参数扫描的Pareto前沿:能耗 vs RMSE}
\label{fig:ode_pareto}
\end{figure}

\subsection{Pontryagin 极小值原理}

给定有限时域优化问题\cite{liberzon2012, kirk2004}:
\begin{equation}
\min_{u(t)} J = \int_0^{T_f} \left[ Q(T - T_{\text{set}})^2 + Ru^2 \right] dt \label{eq:pont_cost}
\end{equation}
约束:$\dot{T} = -k(T - T_a) + u$,$u \in [0, U_{\max}]$。

构造Hamilton量:
\begin{equation}
H = Q(T - T_{\text{set}})^2 + Ru^2 + \lambda[-k(T - T_a) + u] \label{eq:hamiltonian}
\end{equation}

Pontryagin必要条件:
\begin{enumerate}
    \item \textbf{状态方程}:$\dot{T} = -k(T - T_a) + u^*$
    \item \textbf{伴随方程}:$\dot{\lambda} = -2Q(T - T_{\text{set}}) + k\lambda$
    \item \textbf{最优性条件}:$u^* = \text{clip}\!\left(-\frac{\lambda}{2R},\; 0,\; U_{\max}\right)$
\end{enumerate}

边界条件:$T(0) = T_0$,$\lambda(T_f) = 0$(横截性条件),构成两点边值问题(TPBVP)。

\textbf{数值挑战与延续法}:当$R$很小时,$u^*$对$\lambda$非常敏感,BVP求解器难以收敛。我们采用\textbf{延续法}(continuation method):从$R_0 = 2.0$开始,逐步缩小$R$至目标值$0.01$,每步用前一步的解作为初始猜测。延续序列:$R = 2.0 \to 0.667 \to 0.222 \to 0.074 \to 0.025 \to 0.01$,每步均成功收敛。

\begin{figure}[H]
\centering
\includegraphics[width=0.85\textwidth]{ode_pontryagin.png}
\caption{Pontryagin最优控制轨迹:状态、伴随变量和控制信号}
\label{fig:ode_pontryagin}
\end{figure}

\subsection{ODE 模型综合对比}

\begin{table}[H]
\centering
\caption{ODE模型下四种控制策略综合对比}
\label{tab:ode_comparison}
\begin{tabular}{lccccccc}
\toprule
策略 & 能耗 & RMSE & 超调 & 稳定时间 & 切换 & 代价$J$ \\
\midrule
Bang-Bang ($\delta\!=\!0.5$) & 189.4 & 0.527 & 0.500 & 120.0 & 322 & 61.8 \\
PID (默认) & 14.6 & 0.647 & 1.837 & 12.0 & 2 & 50.4 \\
LQR ($Q\!=\!1, R\!=\!0.01$) & 189.6 & 0.442 & 0.000 & 0.7 & 0 & 27.6 \\
Pontryagin ($R\!=\!0.01$) & 188.9 & 0.454 & 0.000 & 1.1 & 1 & 28.7 \\
\textbf{PID (调参后)} & 40.5 & \textbf{0.442} & 0.003 & 0.8 & 0 & \textbf{25.0} \\
\bottomrule
\end{tabular}
\end{table}

\begin{figure}[H]
\centering
\includegraphics[width=\textwidth]{ode_comparison_final.png}
\caption{ODE模型:四种控制策略的温度响应与控制信号对比}
\label{fig:ode_comparison}
\end{figure}

\textbf{关键发现}:

\begin{enumerate}
    \item \textbf{LQR和Pontryagin性能接近}:RMSE分别为0.442和0.454,代价分别为27.6和28.7。两者理论上应趋向相同最优解;微小差异源于LQR是无限时域稳态反馈,Pontryagin是有限时域开环控制。
    \item \textbf{充分调参的PID可匹敌最优控制}:最优PID的$J=25.0$甚至低于LQR的$J=27.6$。这并非悖论——PID通过积分项自适应补偿了系统偏差,而LQR/Pontryagin的$Q$和$R$权重与评价指标不完全匹配。
    \item \textbf{Bang-Bang代价最高}($J=61.8$):频繁切换(322次)不仅增加设备磨损和启动损耗\cite{seem1998},温度持续振荡也使舒适度最差。
\end{enumerate}

\begin{figure}[H]
\centering
\includegraphics[width=0.7\textwidth]{ode_pareto_final.png}
\caption{ODE模型Pareto前沿:能耗 vs RMSE的不可避免权衡}
\label{fig:ode_pareto_final}
\end{figure}

%% ============================================================
\section{一维 PDE 模型}
%% ============================================================

\subsection{从 ODE 到 PDE:引入空间维度}

ODE模型假设房间温度处处均匀,实际中加热器附近温度明显高于远端。沿房间长度$L = 5\,\text{m}$方向建立一维热方程\cite{strikwerda2004}:

\subsubsection*{1D简化的对称性论证}

将三维房间简化为一维模型需要严格的物理论证。我们通过2D仿真定量回答:\textbf{在什么条件下,y方向的温度变化可以忽略?}

定义\textbf{y方向梯度能量比}:$\eta_y = \|\partial T/\partial y\|^2 / (\|\partial T/\partial x\|^2 + \|\partial T/\partial y\|^2)$。当$\eta_y \ll 1$时,温度场近似一维。

\textbf{条件1:加热器必须近似跨越房间全宽}。在标准$5\,\text{m} \times 4\,\text{m}$房间中比较四种加热器配置:
\begin{itemize}
    \item 全宽散热器($\sigma = 2.0\,\text{m}$):$\eta_y = 0.42$
    \item 面板式加热器($\sigma = 0.8\,\text{m}$):$\eta_y = 0.44$
    \item 点式加热器($\sigma = 0.3\,\text{m}$):$\eta_y = 0.61$
    \item 角落加热器:$\eta_y = 0.48$
\end{itemize}

即使全宽散热器也有42\%的梯度能量在y方向,说明1D简化在我们的参数下是一个\textbf{定性近似}而非精确等价。1D模型的主要价值在于捕捉\textbf{沿加热器-恒温器方向}的核心热传输物理——传输延迟和位置效应——而非追求定量精度。

\textbf{条件2:高宽比增大有助于1D近似}。在等面积($20\,\text{m}^2$)的矩形房间中,随着$L_x/L_y$增大,$\eta_y$从0.55(正方形)降至0.25($L_x/L_y = 8$),y方向变化逐渐可忽略。

\begin{figure}[H]
\centering
\includegraphics[width=\textwidth]{symmetry_analysis.png}
\caption{1D简化的对称性分析:(a)$\eta_y$随高宽比变化,(b)1D预测误差,(c)1D与2D平均温度对比,(d-f)不同加热器配置下的2D温度场}
\label{fig:symmetry}
\end{figure}

因此,我们的1D PDE模型应理解为:在\textbf{加热器近似跨越房间全宽}的条件下,对沿房间长度方向温度分布的\textbf{定性描述},用于研究位置效应的基本物理。定量精度需要完整的2D模型。

\begin{equation}
\frac{\partial T}{\partial t} = \alpha \frac{\partial^2 T}{\partial x^2} + S(x, t) \label{eq:pde1d}
\end{equation}
其中$\alpha$为热扩散率,$S(x,t)$为空间分布的热源。

\textbf{加热器模型}:采用高斯空间分布:
\begin{equation}
S(x,t) = \frac{u(t)}{L} \cdot \phi(x), \quad \phi(x) \propto \exp\!\left(-\frac{(x - x_h)^2}{2\sigma^2}\right)
\end{equation}
归一化使$\int_0^L \phi(x)\,dx = L$。

\textbf{Robin边界条件}(模拟墙壁散热):
\begin{equation}
-\alpha \left.\frac{\partial T}{\partial x}\right|_{x=0} = h(T(0,t) - T_a), \quad \alpha \left.\frac{\partial T}{\partial x}\right|_{x=L} = h(T(L,t) - T_a)
\end{equation}

\subsection{数值方法}

采用\textbf{线方法}(Method of Lines):
\begin{enumerate}
    \item 将$[0, L]$等分为$N_x = 51$个网格点,间距$\Delta x = L/(N_x-1)$
    \item 中心差分近似:$\frac{\partial^2 T}{\partial x^2}\big|_{x_i} \approx \frac{T_{i+1} - 2T_i + T_{i-1}}{\Delta x^2}$
    \item Robin BC用虚拟点(ghost point)方法处理
    \item 将空间离散后的$N_x$个ODE交给\texttt{solve\_ivp}(RK45)
\end{enumerate}

\subsection{恒温器位置实验}

加热器固定在$x_h = 0.5\,\text{m}$(左墙附近),扫描恒温器位置(表\ref{tab:1d_thermostat})。

\begin{table}[H]
\centering
\caption{1D PDE 模型:恒温器位置对 LQR 控制器性能的影响}
\label{tab:1d_thermostat}
\begin{tabular}{lccc}
\toprule
恒温器位置 $x_{\text{th}}$ & RMSE ($^\circ$C) & 稳定时间 & 能耗 \\
\midrule
0.5 m(靠近加热器) & 0.506 & 0.5 min & 19.4 \\
1.5 m & 25.869 & 未稳定 & 165.1 \\
2.5 m(中间) & 8.534 & 未稳定 & 629.2 \\
3.5 m & 9.727 & 未稳定 & 900.0 \\
4.5 m(最远端) & 10.465 & 未稳定 & 900.0 \\
\bottomrule
\end{tabular}
\end{table}

\begin{figure}[H]
\centering
\includegraphics[width=\textwidth]{1d_placement_lqr.png}
\caption{1D PDE:不同恒温器位置下LQR控制器的温度场演化}
\label{fig:1d_placement_lqr}
\end{figure}

当恒温器远离加热器时($x_{\text{th}} \geq 2.5\,\text{m}$),热量必须先扩散到恒温器位置。由于纯扩散速度有限且墙壁持续散热,控制器被迫以最大功率持续加热,但仍无法达标。

\subsection{加热器位置实验}

固定恒温器在中心$x_{\text{th}} = 2.5\,\text{m}$,扫描加热器位置(表\ref{tab:1d_heater})。

\begin{table}[H]
\centering
\caption{1D PDE 模型:加热器位置对性能的影响(PID控制,恒温器在中心)}
\label{tab:1d_heater}
\begin{tabular}{lccc}
\toprule
加热器位置 $x_h$ & RMSE ($^\circ$C) & 稳定时间 & 能耗 \\
\midrule
0.5 m(左墙) & 8.545 & 未稳定 & 609.3 \\
1.5 m & 24.178 & 未稳定 & 153.8 \\
\textbf{2.5 m(中心)} & \textbf{1.316} & \textbf{12.4 min} & \textbf{1.6} \\
3.5 m & 24.178 & 未稳定 & 153.8 \\
4.5 m(右墙) & 8.545 & 未稳定 & 609.3 \\
\bottomrule
\end{tabular}
\end{table}

结果呈完美对称性。\textbf{加热器在中心时性能最优},RMSE从8.5$^\circ$C降至1.3$^\circ$C,能耗从609降至1.6——差距超过两个数量级。

\begin{figure}[H]
\centering
\includegraphics[width=0.85\textwidth]{1d_heater_position.png}
\caption{1D PDE:加热器位置对系统性能的影响}
\label{fig:1d_heater}
\end{figure}

\subsection{策略×位置对比矩阵}

\begin{figure}[H]
\centering
\includegraphics[width=\textwidth]{1d_metrics_matrix.png}
\caption{1D PDE:恒温器位置$\times$控制策略的指标矩阵热力图}
\label{fig:1d_metrics_matrix}
\end{figure}

图\ref{fig:1d_metrics_matrix}清晰表明:\textbf{在空间模型中,位置的影响远大于策略的差异}。在同一位置下,三种策略的RMSE差异不超过10\%;而位置变化导致的RMSE差异可达2个数量级。

%% ============================================================
\section{二维 PDE 模型}
%% ============================================================

\subsection{数学模型}

将房间建模为$L_x \times L_y = 5\,\text{m} \times 4\,\text{m}$的矩形域\cite{strikwerda2004}:
\begin{equation}
\frac{\partial T}{\partial t} = \alpha \left(\frac{\partial^2 T}{\partial x^2} + \frac{\partial^2 T}{\partial y^2}\right) + S(x, y, t)
\end{equation}
四面墙壁均采用Robin边界条件。加热器为二维高斯分布:
\begin{equation}
S(x,y,t) = \frac{u(t)}{L_x L_y} \cdot \psi(x,y), \quad \psi \propto \exp\!\left(-\frac{(x-x_h)^2 + (y-y_h)^2}{2\sigma^2}\right)
\end{equation}

在$21 \times 17$的网格上用二维中心差分离散Laplacian,共357个ODE交给\texttt{solve\_ivp}。

\subsection{恒温器与加热器位置实验}

\begin{table}[H]
\centering
\caption{2D PDE 模型:加热器位置对性能的影响(恒温器在中心)}
\label{tab:2d_heater}
\begin{tabular}{lccc}
\toprule
加热器位置 & RMSE ($^\circ$C) & 能耗 & 稳定时间 \\
\midrule
角落 (0.5, 0.5) & 9.811 & 450.0 & 未稳定 \\
墙壁中部 (0.5, 2.0) & 8.824 & 450.0 & 未稳定 \\
\textbf{中心 (2.5, 2.0)} & \textbf{1.187} & \textbf{20.9} & \textbf{4.5 min} \\
对面墙壁 (4.5, 2.0) & 8.824 & 450.0 & 未稳定 \\
\bottomrule
\end{tabular}
\end{table}

\begin{figure}[H]
\centering
\includegraphics[width=\textwidth]{2d_heater_position.png}
\caption{2D PDE:不同加热器位置下的温度场对比}
\label{fig:2d_heater}
\end{figure}

\textbf{加热器放在中心是唯一能让系统稳定的配置},与1D结论一致。在2D中更加显著——角落放置比墙壁中部更差,因为热量需要向两个方向扩散。

\subsection{空间均匀性分析}

\begin{figure}[H]
\centering
\includegraphics[width=\textwidth]{2d_uniformity.png}
\caption{2D PDE:空间温度均匀性分析——恒温器读数 vs 空间平均温度}
\label{fig:2d_uniformity}
\end{figure}

即使恒温器读数达到$T_{\text{set}}$,房间内仍有部分区域温度显著偏低。单点传感器无法代表全房间的热舒适度——这是"恒温器放在哪"问题的物理本质。

\subsection{不同房间形状的影响}

项目问题文件建议研究\emph{different shaped 2D rooms}。我们通过在有限差分网格上引入\textbf{域掩模}(domain mask),将非矩形区域(如L形房间)作为矩形网格的子集进行仿真。掩模外的格点保持为$T_a$,掩模内外边界采用与外墙相同的Robin边界条件。

\subsubsection*{不同高宽比的矩形房间}

在保持面积恒定($20\,\text{m}^2$)的条件下,比较五种高宽比($L_x/L_y = 1, 1.5, 2, 3, 5$)的矩形房间。加热器固定在左墙附近,恒温器在中心。

结果表明:随着房间变窄变长,温度梯度沿长轴增大。对于$L_x/L_y \geq 3$的走廊形房间,加热器一端的温度远高于远端,空间温度标准差显著增加。这意味着\textbf{走廊形房间的控制更加困难},对恒温器位置更敏感。

\subsubsection*{L形房间}

L形房间通过从矩形($6\,\text{m} \times 6\,\text{m}$)右上角切除一个矩形区域得到。我们比较了两种切除比例(50\%和33\%)与完整矩形。

\begin{figure}[H]
\centering
\includegraphics[width=\textwidth]{room_shapes_comparison.png}
\caption{不同房间形状下的温度场对比:上排为不同高宽比矩形,下排为矩形与L形}
\label{fig:room_shapes}
\end{figure}

\textbf{L形房间的核心问题}:加热器位于L形一翼时,另一翼成为"冷角"(cold corner),因为热量必须绕过L形拐角才能到达,扩散路径显著增长。这是1D模型\textbf{无法捕捉}的几何效应。

\subsubsection*{L形房间的恒温器最优放置}

在L形房间内扫描恒温器位置,发现:\textbf{最优恒温器位置不在几何中心,而在L形两翼的交界处附近}(图\ref{fig:room_shapes_placement})。直觉解释:交界处是两翼温度的"汇合点",此处的读数最接近全房间的空间平均。

\begin{figure}[H]
\centering
\includegraphics[width=\textwidth]{room_shapes_placement.png}
\caption{(a) L形房间恒温器位置优化热力图——最优位置在两翼交界处,(b) 不同房间形状的温度不均匀性对比}
\label{fig:room_shapes_placement}
\end{figure}

这一发现对实际工程有重要意义:对于非矩形房间(如公寓中的L形客厅),恒温器应放在走廊和主要居住区的过渡位置,而非简单地放在房间中心或墙壁上。

%% ============================================================
\section{对流效应与物理模型改进}
%% ============================================================

\subsection{纯扩散模型的局限}

前面PDE实验中,大量配置显示"未稳定"($t_{\text{settle}} = \infty$)。这是否意味着大多数供暖系统在物理上不可行?答案是否定的。

根本原因是\textbf{扩散时间尺度}:
\begin{equation}
\tau_{\text{diff}} = \frac{L^2}{\alpha} = \frac{25}{0.01} = 2500\,\text{min} \approx 42\,\text{h}
\end{equation}
远大于仿真时间120分钟。在纯扩散模型中,热量从加热器扩散到5米外的墙壁需要数十小时——这显然不符合实际经验。

\subsection{对流增强扩散}

实际房间中空气存在\textbf{自然对流}(温度梯度驱动的浮力流)和\textbf{强制对流}(风扇、空调送风)。对流效应可通过增大有效扩散率$\alpha_{\text{eff}}$来建模:
\begin{equation}
\alpha_{\text{eff}} = \alpha_{\text{diffusion}} + \alpha_{\text{convection}} \gg \alpha_{\text{diffusion}}
\end{equation}

我们对$\alpha \in \{0.005, 0.01, 0.02, 0.05, 0.1, 0.2, 0.5\}\,\text{m}^2/\text{min}$进行了系统性灵敏度分析。

\begin{figure}[H]
\centering
\includegraphics[width=\textwidth]{adv_convection_1d.png}
\caption{对流效应:不同有效扩散率$\alpha$下的1D温度场演化}
\label{fig:convection}
\end{figure}

\begin{figure}[H]
\centering
\includegraphics[width=0.8\textwidth]{adv_alpha_sensitivity.png}
\caption{$\alpha$灵敏度分析:RMSE和稳定时间随$\alpha$的变化}
\label{fig:alpha_sensitivity}
\end{figure}

\begin{table}[H]
\centering
\caption{有效扩散率$\alpha$灵敏度分析(1D PDE, PID控制, 加热器$x_h=0.5$m, 恒温器$x_{\text{th}}=2.5$m)}
\label{tab:convection}
\begin{tabular}{lccc}
\toprule
$\alpha$ (m$^2$/min) & RMSE ($^\circ$C) & 稳定时间 & 能耗 \\
\midrule
0.005 & 34.91 & 未稳定 & 900.0 \\
0.01(纯扩散) & 8.55 & 未稳定 & 609.3 \\
0.02 & 7.51 & 未稳定 & 262.7 \\
0.05 & 6.62 & 未稳定 & 30.9 \\
\textbf{0.1(弱对流)} & \textbf{6.25} & \textbf{未稳定} & \textbf{8.3} \\
0.2(中等对流) & 3.01 & 未稳定 & 3.2 \\
\textbf{0.5(强对流/风扇)} & \textbf{1.87} & \textbf{33.1 min} & \textbf{1.8} \\
\bottomrule
\end{tabular}
\end{table}

\textbf{核心洞察}:$\alpha$从0.01增大到0.1时(引入弱对流),RMSE从8.55$^\circ$C降至6.25$^\circ$C,能耗从609降至8.3;当$\alpha = 0.5$(强制对流/风扇辅助)时,RMSE进一步降至1.87$^\circ$C并可在33分钟内稳定。这解释了为何实际房间中供暖系统可以在合理时间内达到设定温度。

\subsection{对流条件下的控制策略对比}

在$\alpha = 0.1$的对流增强模型下比较各控制策略:

\begin{figure}[H]
\centering
\includegraphics[width=\textwidth]{adv_1d_convection_controllers.png}
\caption{对流增强1D PDE下的控制策略对比($\alpha = 0.1\,\text{m}^2/\text{min}$)}
\label{fig:convection_controllers}
\end{figure}

即使引入对流,控制策略间的差异仍然较小(LQR和PID的RMSE差距$<3\%$),进一步印证"位置$>$策略"的结论。

%% ============================================================
\section{加热器-恒温器联合位置优化}
%% ============================================================

\subsection{1D 联合优化}

前面的实验分别固定了加热器或恒温器位置。现在我们进行\textbf{联合优化}:在$\alpha = 0.1$条件下,对加热器位置$x_h$和恒温器位置$x_{\text{th}}$组成的$10\times 10$网格进行穷举搜索。

\begin{figure}[H]
\centering
\includegraphics[width=0.75\textwidth]{adv_optimal_placement_1d.png}
\caption{1D联合位置优化:RMSE在加热器$\times$恒温器位置空间的分布}
\label{fig:optimal_1d}
\end{figure}

最优配置为加热器$x_h = 0.25\,\text{m}$、恒温器$x_{\text{th}} = 0.75\,\text{m}$(两者靠近),RMSE = 0.848$^\circ$C。这再次证明:\textbf{加热器与恒温器的相对距离是第一决定因素}。当两者靠近时,控制器能精确感知并调节局部温度;当两者远离时,传输延迟使反馈控制退化。

\subsection{2D 联合优化}

在$5\,\text{m} \times 4\,\text{m}$房间的$15\times 15$网格上进行2D联合优化:

\begin{figure}[H]
\centering
\includegraphics[width=0.75\textwidth]{adv_optimal_placement_2d.png}
\caption{2D联合位置优化:最优加热器与恒温器位置}
\label{fig:optimal_2d}
\end{figure}

最优配置为加热器(4.5, 0.5)、恒温器(3.5, 0.5),RMSE = 2.04$^\circ$C。两者相距约1m,紧密配对。

%% ============================================================
\section{多房间建筑模型}
%% ============================================================

\subsection{模型建立}

项目要求制定建筑整体策略。我们建立$N$个房间线性排列的多房间耦合模型:

\begin{equation}
\frac{dT_i}{dt} = -k_{\text{ext}}(T_i - T_a) \cdot \mathbf{1}_{\{i\in\text{ext}\}} - k_{\text{int}}\sum_{j\in\text{adj}(i)}(T_i - T_j) + u_i(t)
\end{equation}

房间布局:$[\text{ext}\;|\;\text{room 0}\;|\;\text{int}\;|\;\text{room 1}\;|\;\cdots\;|\;\text{room }N\!-\!1\;|\;\text{ext}]$

其中$k_{\text{ext}} = 0.1\,\text{min}^{-1}$为外墙冷却常数,$k_{\text{int}} = 0.05\,\text{min}^{-1}$为内墙耦合常数。矩阵形式:
\begin{equation}
\dot{\mathbf{T}} = A\mathbf{T} + \mathbf{u} + \mathbf{c}
\end{equation}
其中$A$编码了外墙散热和内墙耦合的拓扑结构。

\subsection{内墙耦合效应}

\begin{figure}[H]
\centering
\includegraphics[width=\textwidth]{adv_building_coupling.png}
\caption{5房间建筑:不同内墙耦合强度$k_{\text{int}}$下各房间的温度响应}
\label{fig:building_coupling}
\end{figure}

随着$k_{\text{int}}$增大,房间间温度趋于均匀,但也意味着热量从被加热房间流失到邻室。外墙房间(room 0和room 4)由于同时面对室外散热和内部耦合,温度始终低于内部房间。

\subsection{建筑级控制策略对比}

我们在5房间建筑上比较了五种策略:

\begin{enumerate}
    \item \textbf{Uniform PID}:所有房间使用相同PID参数
    \item \textbf{Adaptive PID}:外墙房间使用更高增益($K_p$加倍,$K_i$加倍)
    \item \textbf{LQR}:基于完整耦合模型的最优反馈
    \item \textbf{Exterior-only}:仅加热外墙房间,依靠内墙耦合温暖内部
    \item \textbf{Exterior + Centre}:仅加热外墙和中心房间
\end{enumerate}

\begin{figure}[H]
\centering
\includegraphics[width=\textwidth]{adv_building_strategies.png}
\caption{建筑级控制策略对比:5种策略下各房间温度轨迹}
\label{fig:building_strategies}
\end{figure}

\begin{table}[H]
\centering
\caption{建筑级控制策略量化对比}
\label{tab:building_strategies}
\begin{tabular}{lcccc}
\toprule
策略 & 平均 RMSE & 最差房间 RMSE & 总能耗 & 评价 \\
\midrule
Uniform PID & 0.496 & 0.696 & 55.3 & 基线 \\
\textbf{Adaptive PID} & \textbf{0.484} & \textbf{0.486} & 58.8 & \textbf{最均匀} \\
LQR & 0.496 & 0.696 & 56.5 & 理论最优 \\
Exterior-only & 2.440 & 5.990 & 25.0 & 最节能 \\
Exterior + Centre & 0.917 & 2.156 & 46.1 & 平衡方案 \\
\bottomrule
\end{tabular}
\end{table}

\begin{figure}[H]
\centering
\includegraphics[width=0.75\textwidth]{adv_building_pareto.png}
\caption{建筑级策略的Pareto前沿:总能耗 vs 平均RMSE}
\label{fig:building_pareto}
\end{figure}

\textbf{关键发现}:
\begin{enumerate}
    \item \textbf{Adaptive PID最均匀}:通过对外墙房间增大增益,最差房间RMSE从0.696$^\circ$C降至0.486$^\circ$C,房间间温差减小。
    \item \textbf{Exterior+Centre是优秀的Pareto方案}:仅加热3个房间(节省40\%能耗),平均RMSE为0.917$^\circ$C,远好于仅加热外墙(RMSE=2.44$^\circ$C)。
    \item \textbf{LQR的优势不明显}:在多房间场景中,LQR与Uniform PID性能几乎相同,因为房间间耦合较弱时,局部反馈已足够好。
\end{enumerate}

%% ============================================================
\section{Pontryagin 最优控制的数值方法}
%% ============================================================

\subsection{BVP 的数值困难}

当$R$很小时,$u^* = \text{clip}(-\lambda/(2R), 0, U_{\max})$对$\lambda$非常敏感,伴随方程变得\textbf{刚性}——$\lambda$在短时间内急剧变化。

\subsection{延续法的实现}

参数延续的关键思想:将困难问题转化为一系列容易问题的序列。每一步的解为下一步提供良好的初始猜测,使得Newton迭代从一开始就在收敛半径内。

对$R = 0.01$,完整延续序列为:
\[
R = 2.0 \to 0.667 \to 0.222 \to 0.074 \to 0.025 \to 0.01
\]
每步均在800-870个网格节点上成功收敛。没有延续法时,直接求解$R = 0.01$的BVP完全不收敛。

\subsection{与 LQR 的理论关系}

在无约束和无限时域极限下,Pontryagin解应与LQR解重合\cite{astrom2021, liberzon2012}。数值结果的微小差异($J_{\text{LQR}} = 27.6$ vs $J_{\text{Pont}} = 28.7$)来自:有限时域终端效应($\lambda \to 0$时控制力减弱)、控制约束激活、以及BVP容差。

%% ============================================================
\section{创新方案:解决恒温器位置敏感性问题}
%% ============================================================

前面的实验揭示了一个核心问题:单点恒温器远离加热器时,控制性能急剧恶化。但这并不意味着恒温器必须放在加热器旁边——问题的根源是\textbf{单点测量无法代表空间平均温度}。本节提出四种创新解决方案。

\subsection{方案概述}

\begin{enumerate}
    \item \textbf{设定值补偿}:根据模型预测的稳态温度梯度,调整恒温器处的本地设定值。如果恒温器在加热器附近(读数偏高),则降低本地目标;反之则升高。
    \item \textbf{观测器反馈}:用稳态模型估算"恒温器读数$\to$空间平均温度"的偏差,作为单传感器的模型补偿项。PID控制器作用于估算的平均温度而非原始读数。
    \item \textbf{双传感器加权}:在房间两端各放一个温度传感器,取加权平均作为反馈信号:$T_{\text{fb}} = w \cdot T_{\text{near}} + (1-w) \cdot T_{\text{far}}$。无需物理模型。
    \item \textbf{脉冲加热}:交替进行"加热期"(PID正常工作)和"扩散期"(加热器关闭,让热量均匀化)。在扩散期读取恒温器,获得更具代表性的温度。
\end{enumerate}

\subsection{实验结果}

在对流增强1D PDE模型($\alpha_{\text{eff}} = 0.1\,\text{m}^2/\text{min}$)上,加热器固定于$x_h = 0.5\,\text{m}$,比较各方案在不同恒温器位置下的性能。评价指标为\textbf{空间平均温度的RMSE}(而非恒温器本地读数)。

\begin{table}[H]
\centering
\caption{创新方案 vs 基线PID:不同恒温器位置下的空间平均RMSE ($^\circ$C)}
\label{tab:solutions}
\begin{tabular}{lccccc}
\toprule
方案 & $x_{\text{th}}=0.5$ & $1.5$ & $2.5$ & $3.5$ & $4.5$ \\
\midrule
基线 PID & 4.27 & 3.25 & 6.31 & 16.68 & 29.66 \\
设定值补偿 & 9.42 & 6.49 & 6.03 & 20.61 & 38.72 \\
观测器反馈 & \textbf{2.15} & \textbf{1.23} & 6.74 & 13.08 & 18.66 \\
\textbf{双传感器} & 1.96 & 1.96 & \textbf{1.96} & \textbf{1.96} & \textbf{1.96} \\
脉冲加热 & 2.56 & 2.42 & 5.16 & 13.32 & 22.60 \\
\bottomrule
\end{tabular}
\end{table}

\begin{figure}[H]
\centering
\includegraphics[width=\textwidth]{placement_solutions_comparison.png}
\caption{创新方案对比:空间平均RMSE和能耗随恒温器位置的变化}
\label{fig:solutions_comparison}
\end{figure}

\subsection{核心创新:双传感器加权反馈}

双传感器方案的最突出优势是\textbf{位置无关性}——无论名义恒温器在哪,RMSE恒为$1.96\,^\circ\text{C}$。这是因为反馈信号来自两个固定位置(近端$x=0.5\,\text{m}$和远端$x=4.5\,\text{m}$)的加权平均,直接近似了空间平均温度。

\textbf{权重优化}:扫描$w_{\text{near}} \in [0, 1]$,最优权重为$w^* = 0.45$(近乎等权),对应最低RMSE = $1.60\,^\circ\text{C}$。

\begin{figure}[H]
\centering
\includegraphics[width=0.9\textwidth]{placement_dual_sensor_optim.png}
\caption{双传感器权重优化:RMSE随近端传感器权重的变化及能耗-RMSE Pareto}
\label{fig:dual_sensor_optim}
\end{figure}

\textbf{传感器位置优化}:

\begin{table}[H]
\centering
\caption{双传感器布局方案对比}
\label{tab:dual_sensor_placement}
\begin{tabular}{lcc}
\toprule
传感器位置 & RMSE ($^\circ$C) & 能耗 \\
\midrule
近+远 (0.5m, 4.5m) & 1.96 & 80.3 \\
近+中 (0.5m, 2.5m) & 2.76 & 116.0 \\
\textbf{四分位 (1.25m, 3.75m)} & \textbf{1.27} & 174.1 \\
三分位 (1.67m, 3.33m) & 3.99 & 181.3 \\
\bottomrule
\end{tabular}
\end{table}

四分位布局($x = L/4$和$x = 3L/4$)达到最低RMSE = $1.27\,^\circ\text{C}$。这与数值积分中的\textbf{高斯求积}思想一致:两点高斯求积的最优节点恰好在区间的$1/4$和$3/4$处,能精确积分三次多项式——这里的温度稳态分布近似为二次函数,两个四分位传感器恰好能最佳近似其空间平均值。

\begin{figure}[H]
\centering
\includegraphics[width=\textwidth]{placement_solutions_profiles.png}
\caption{各方案在$t=60\,\text{min}$时的空间温度分布}
\label{fig:solutions_profiles}
\end{figure}

\begin{figure}[H]
\centering
\includegraphics[width=0.85\textwidth]{placement_solutions_improvement.png}
\caption{各创新方案相对于基线PID的RMSE改善百分比}
\label{fig:solutions_improvement}
\end{figure}

\subsection{方案比较总结}

\begin{table}[H]
\centering
\caption{四种创新方案的特性对比}
\begin{tabular}{lcccc}
\toprule
方案 & 需要模型? & 额外硬件? & 位置鲁棒性 & 最佳适用场景 \\
\midrule
设定值补偿 & 是 & 否 & 差 & 恒温器在加热器附近 \\
观测器反馈 & 是 & 否 & 中 & 已知模型参数 \\
\textbf{双传感器} & \textbf{否} & 1个传感器 & \textbf{优} & \textbf{通用场景} \\
脉冲加热 & 否 & 否 & 中 & 无法加装传感器 \\
\bottomrule
\end{tabular}
\end{table}

双传感器方案以最低的复杂度(仅需多加一个廉价温度传感器)实现了最优且位置无关的性能,是本项目提出的\textbf{核心创新方案}。

%% ============================================================
\section{讨论}
%% ============================================================

\subsection{从 0D 到 2D 的认知递进}

本项目最重要的方法论贡献是展示了逐步复杂化建模的价值:

\begin{table}[H]
\centering
\caption{模型维度递进带来的新洞察}
\begin{tabular}{lll}
\toprule
模型维度 & 新增物理效应 & 新增洞察 \\
\midrule
ODE (0D) & — & 策略优劣:LQR/Pontryagin $>$ PID $>$ Bang-Bang \\
1D PDE & 空间扩散、位置效应 & 位置 $>$ 策略 \\
2D PDE & 二维扩散、角落效应 & 中心放置最优;均匀性是关键挑战 \\
对流增强 & 自然/强制对流 & 纯扩散严重低估实际热响应速度 \\
多房间 & 内墙耦合 & 自适应增益优于统一参数 \\
\bottomrule
\end{tabular}
\end{table}

每一步复杂化都揭示了新的物理机制,而非简单的精度提升。这体现了应用数学建模的核心思想:\emph{``所有模型都是错的,但有些是有用的''}(George Box)。

\subsection{核心创新:位置$>$策略}

在集总参数(ODE)模型中,控制策略的选择是唯一的优化变量,LQR/Pontryagin可比Bang-Bang降低代价55\%。然而,在PDE模型中,\textbf{加热器从墙壁移至中心,RMSE降低85\%,能耗降低99.7\%}——这一改善远超任何控制策略优化。

这一发现的实际工程意义是:在设计供暖系统时,\textbf{先优化设备布局,再选择控制策略}。

\subsection{对流效应的物理意义}

纯扩散模型($\alpha = 0.01\,\text{m}^2/\text{min}$)给出不切实际的慢响应(扩散时间尺度42小时),引入有效对流扩散率后模型行为与实际经验一致。这提醒我们:\textbf{简化模型必须选择正确的物理尺度}。

\subsection{控制策略的适用场景}

\begin{table}[H]
\centering
\begin{tabular}{lll}
\toprule
场景 & 推荐策略 & 理由 \\
\midrule
简单家用恒温器 & Bang-Bang ($\delta = 0.5\,^\circ$C) & 实现最简单,鲁棒性好 \\
精密温控(实验室) & PID(充分调参) & 零超调、快速稳定 \\
已知精确模型 & LQR & 数学最优,自动权衡 \\
有限时域优化 & Pontryagin & 理论最严谨 \\
多房间建筑 & Adaptive PID & 无需全局模型,适应性强 \\
\bottomrule
\end{tabular}
\end{table}

\subsection{Zeno效应的理论意义}

Zeno效应分析揭示了Bang-Bang控制的一个根本性限制:作为混合动力系统\cite{goebel2012, zhang2001},当滞回带$\delta = 0$时,系统在$T = T_{\text{set}}$处存在Zeno点——两个模式的向量场方向相反($f_{\text{ON}} > 0$, $f_{\text{OFF}} < 0$),导致理论上无限次切换。Filippov正则化\cite{filippov1988}给出的滑动模式$u_{\text{eq}} = k(T_{\text{set}} - T_a)$恰好是维持设定温度的精确功率,这一解析结果与数值仿真完全一致。

从工程角度,Zeno效应解释了为什么实际恒温器\textbf{必须}有滞回带:
\begin{itemize}
    \item $\delta$过小:切换频率$\sim 1/\delta$导致继电器寿命缩短、启动暂态能耗增加;
    \item $\delta$过大:温度波动增大,舒适度下降;
    \item 引入切换能耗惩罚$c_s$后,存在一个最优$\delta^*$使总能耗最小(图\ref{fig:zeno_convergence}b)。
\end{itemize}

\subsection{房间几何形状的影响}

L形房间实验揭示了一个1D模型无法捕捉的几何效应:当加热器在L形的一翼时,另一翼因扩散路径增长而成为持续性冷区。此时恒温器的最优位置不在几何中心,而在两翼交界处——该位置的温度读数最接近全房间平均值。

这一发现对实际建筑设计具有直接工程意义:对于非矩形户型(常见于老式公寓和翻新建筑),恒温器放置需要考虑房间几何的连通性,而非仅依据面积中心。

\subsection{局限与未来工作}

\begin{enumerate}
    \item \textbf{模型简化}:实际房间有家具、窗户(不同传热系数)、人体热源等。我们的参数验证\cite{cibse2015, ashrae2021}表明模型对轻型建筑物理合理,但重型建筑需要更大的$\tau$。
    \item \textbf{控制器假设}:LQR和Pontryagin基于完美模型知识;鲁棒控制($H_\infty$)或自适应控制是自然延伸。
    \item \textbf{加热器模型}:真实暖气片有热响应时间延迟。
    \item \textbf{MPC}:模型预测控制可在线处理约束和扰动,是LQR/Pontryagin的现代工程延伸。
    \item \textbf{三维效应}:我们的2D模型忽略了垂直方向的温度分层(热空气上升),这在高天花板房间中可能显著。
    \item \textbf{更复杂的房间形状}:本文仅考虑了L形,T形、U形等更复杂拓扑结构的研究是自然推广。
\end{enumerate}

%% ============================================================
\section{结论}
%% ============================================================

本项目通过从ODE到2D PDE再到多房间建筑的递进建模,以及四种控制策略的系统比较,得出以下核心结论:

\begin{enumerate}
    \item \textbf{在集总参数模型中},LQR和Pontryagin达到最低代价($J \approx 27$-$28$),但充分调参的PID可达到$J = 25$,说明"最优控制$\neq$最优性能"——模型与评价准则的匹配同样重要。

    \item \textbf{引入空间维度后},加热器和恒温器的\textbf{位置}成为控制性能的主导因素。加热器从墙壁移至中心,RMSE降低85\%,能耗降低99.7\%。这一改善远超任何控制策略优化。

    \item \textbf{对流效应是PDE模型可控性的关键}:纯扩散模型给出42小时的特征时间尺度,引入弱对流($\alpha = 0.1$)后系统行为合理,强对流($\alpha = 0.5$)下可在33分钟稳定。

    \item \textbf{在多房间建筑中},外墙房间使用更高增益的自适应PID策略优于统一参数策略;选择性加热外墙+中心房间可节省40\%能耗,同时保持可接受的舒适度。

    \item \textbf{双传感器加权反馈消除位置敏感性}:在房间四分位点($L/4$和$3L/4$)各放一个传感器,取近等权平均作为反馈信号,可将RMSE降至$1.27\,^\circ\text{C}$且\textbf{完全不依赖传感器位置}。这一方案无需物理模型,仅需增加一个廉价传感器,是本项目提出的核心创新。其最优性与高斯求积理论一致:两点高斯节点恰在$[0,L]$的四分位处。

    \item \textbf{实际工程建议}:(a)将加热器置于房间中心;(b)使用两个传感器(四分位布局)替代单点恒温器;(c)在此基础上,PID控制器(充分调参)即可达到接近最优的性能。

    \item \textbf{Zeno效应在$\delta = 0$时确实存在}:数学分析证明当滞回带为零时,切换次数$N \sim 1/\delta \to \infty$。Filippov正则化给出等效连续控制$u_{\text{eq}} = k(T_{\text{set}} - T_a)$。引入切换能耗惩罚后,存在最优$\delta^*$平衡舒适度与设备寿命。

    \item \textbf{房间形状影响控制难度}:L形房间的拐角区域热传输路径增长,产生持续性冷区。恒温器最优位置在两翼交界处而非几何中心。走廊形房间(高宽比$>3$)沿长轴梯度显著增大。

    \item \textbf{模型参数经实际数据验证}:热时间常数$\tau = 10$分钟对应CIBSE轻型建筑分类,等效加热功率15 kW与中型中央供暖系统一致,有效热扩散率$\alpha = 0.01\,\text{m}^2/\text{min}$包含约8倍对流增强因子,Biot数$\text{Bi} = 2.5$表明内部温度梯度不可忽略(PDE必要)。
\end{enumerate}

%% ============================================================
\section*{参考文献}
%% ============================================================

\begin{thebibliography}{99}

\bibitem{boyce2021}
Boyce WE, DiPrima RC, Meade DB.
\emph{Elementary Differential Equations and Boundary Value Problems}. 12th~ed. Wiley; 2021.

\bibitem{strogatz2024}
Strogatz SH.
\emph{Nonlinear Dynamics and Chaos}. 3rd~ed. CRC Press; 2024.

\bibitem{astrom2021}
\AA str\"om KJ, Murray RM.
\emph{Feedback Systems: An Introduction for Scientists and Engineers}. 2nd~ed. Princeton University Press; 2021.

\bibitem{goebel2012}
Goebel R, Sanfelice RG, Teel AR.
\emph{Hybrid Dynamical Systems: Modeling, Stability, and Robustness}. Princeton University Press; 2012.

\bibitem{zhang2001}
Zhang J, Johansson KH, Lygeros J, Sastry SS.
Zeno hybrid systems.
\emph{Int J Robust Nonlinear Control}. 2001;11(5):435--451.

\bibitem{lygeros2003}
Lygeros J, Johansson KH, Simic SN, Zhang J, Sastry SS.
Dynamical properties of hybrid automata.
\emph{IEEE Trans Autom Control}. 2003;48(1):2--17.

\bibitem{strikwerda2004}
Strikwerda JC.
\emph{Finite Difference Schemes and Partial Differential Equations}. 2nd~ed. SIAM; 2004.

\bibitem{blasco2012}
Blasco C, Monreal J, Benitez I, Lluna A.
Modelling and PID control of HVAC system.
In: \emph{Trends in Practical Applications of Agents and Multiagent Systems}. Springer; 2012. p.~365--374.

\bibitem{anderson1990}
Anderson BDO, Moore JB.
\emph{Optimal Control: Linear Quadratic Methods}. Prentice-Hall; 1990. (Dover reprint, 2007.)

\bibitem{liberzon2012}
Liberzon D.
\emph{Calculus of Variations and Optimal Control Theory: A Concise Introduction}. Princeton University Press; 2012.

\bibitem{kirk2004}
Kirk DE.
\emph{Optimal Control Theory: An Introduction}. Dover; 2004.

\bibitem{seem1998}
Seem JE.
Duct leakage impacts on short cycling in HVAC systems.
\emph{ASHRAE Journal}. 1998;40(8).

\bibitem{ashrae2021}
ASHRAE.
\emph{HVAC Systems and Equipment Handbook}. Ch.~47; 2021.

\bibitem{cibse2015}
CIBSE.
\emph{Guide A: Environmental Design}. 8th~ed. Chartered Institution of Building Services Engineers; 2015.

\bibitem{bacher2011}
Bacher P, Madsen H.
Identifying suitable models for the heat dynamics of buildings.
\emph{Energy and Buildings}. 2011;43(7):1511--1522.

\bibitem{filippov1988}
Filippov AF.
\emph{Differential Equations with Discontinuous Righthand Sides}. Kluwer; 1988.

\end{thebibliography}

%% ============================================================
\appendix
\section{物理参数表}
%% ============================================================

\begin{table}[H]
\centering
\begin{tabular}{llll}
\toprule
参数 & 符号 & 值 & 单位 \\
\midrule
室外温度 & $T_a$ & 5 & $^\circ$C \\
初始室温 & $T_0$ & 10 & $^\circ$C \\
设定温度 & $T_{\text{set}}$ & 20 & $^\circ$C \\
最大加热率 & $U_{\max}$ & 15 & $^\circ$C/min \\
冷却常数 & $k$ & 0.1 & min$^{-1}$ \\
热扩散率 & $\alpha$ & 0.01 & m$^2$/min \\
有效扩散率(对流) & $\alpha_{\text{eff}}$ & 0.1--0.5 & m$^2$/min \\
墙壁传热系数 & $h$ & 0.5 & m$^{-1}$ \\
房间长度 & $L_x$ & 5 & m \\
房间宽度 & $L_y$ & 4 & m \\
滞回带半宽 & $\delta$ & 0.5 & $^\circ$C \\
仿真总时间 & $T_f$ & 120 & min \\
外墙冷却常数 & $k_{\text{ext}}$ & 0.1 & min$^{-1}$ \\
内墙耦合常数 & $k_{\text{int}}$ & 0.05 & min$^{-1}$ \\
\bottomrule
\end{tabular}
\end{table}

\section{PID 参数搜索结果(前5名)}

\begin{table}[H]
\centering
\begin{tabular}{cccccc}
\toprule
排名 & $K_p$ & $K_i$ & $K_d$ & RMSE & 代价 $J$ \\
\midrule
1 & 8.0 & 1.0 & 0.0 & 0.442 & 25.01 \\
2 & 8.0 & 0.5 & 0.0 & 0.443 & 25.33 \\
3 & 8.0 & 1.0 & 0.5 & 0.443 & 25.38 \\
4 & 4.0 & 1.0 & 0.0 & 0.480 & 26.98 \\
5 & 8.0 & 0.5 & 0.5 & 0.444 & 25.61 \\
\bottomrule
\end{tabular}
\end{table}

高$K_p$(快速响应)和适度$K_i$(消除稳态误差)是关键,$K_d$对此一阶系统贡献甚微。

\end{document}
