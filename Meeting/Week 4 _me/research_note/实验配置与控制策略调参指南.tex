\documentclass[11pt,a4paper]{article}

% 禁用 PDF tagging
\ExplSyntaxOn
\bool_if_exist:NT \g__tag_active_bool { \bool_gset_false:N \g__tag_active_bool }
\bool_if_exist:NT \g__tag_active_struct_dest_bool { \bool_gset_false:N \g__tag_active_struct_dest_bool }
\ExplSyntaxOff

\usepackage[UTF8]{ctex}
\usepackage{amsmath,amssymb,amsthm,mathtools}
\usepackage{geometry}
\usepackage{tikz}
\usetikzlibrary{arrows.meta,positioning,decorations.pathreplacing,calc,patterns}
\usepackage{pgfplots}
\pgfplotsset{compat=1.18}
\usepackage{booktabs}
\usepackage{enumitem}
\usepackage{xcolor}
\usepackage{tcolorbox}
\tcbuselibrary{breakable}
\usepackage{fancyhdr}
\usepackage{titlesec}
\usepackage{array}
\usepackage{tabularx}
\usepackage{makecell}
\usepackage{multirow}
\usepackage{listings}

\usepackage[backend=biber, style=authoryear, maxnames=2, natbib=true]{biblatex}
\addbibresource{references_guide.bib}

\usepackage{hyperref}
\hypersetup{colorlinks=true, linkcolor=blue!60!black, citecolor=green!50!black, urlcolor=cyan!60!black}

\geometry{left=2.2cm, right=2.2cm, top=2.2cm, bottom=2.2cm}

% 颜色定义
\definecolor{defcolor}{RGB}{0,100,150}
\definecolor{thmcolor}{RGB}{150,50,0}
\definecolor{keycolor}{RGB}{200,0,50}
\definecolor{notecolor}{RGB}{0,120,60}
\definecolor{warncolor}{RGB}{180,80,0}
\definecolor{comparecolor}{RGB}{100,0,150}
\definecolor{codegreen}{rgb}{0.1,0.5,0.1}
\definecolor{codebg}{RGB}{245,245,250}

% 定理环境
\theoremstyle{definition}
\newtheorem{definition}{定义}[section]
\newtheorem{theorem}{定理}[section]
\newtheorem{remark}{注记}[section]
\newtheorem{proposition}{命题}[section]

% tcolorbox 环境
\newtcolorbox{keybox}[1][]{colback=red!5!white, colframe=red!60!black, title=#1, fonttitle=\bfseries, breakable}
\newtcolorbox{notebox}[1][]{colback=green!5!white, colframe=green!60!black, title=#1, fonttitle=\bfseries, breakable}
\newtcolorbox{derivbox}[1][]{colback=blue!5!white, colframe=blue!60!black, title=#1, fonttitle=\bfseries, breakable}
\newtcolorbox{warnbox}[1][]{colback=orange!5!white, colframe=orange!60!black, title=#1, fonttitle=\bfseries, breakable}
\newtcolorbox{comparebox}[1][]{colback=violet!5!white, colframe=violet!60!black, title=#1, fonttitle=\bfseries, breakable}
\newtcolorbox{flowbox}[1][]{colback=cyan!5!white, colframe=cyan!60!black, title=#1, fonttitle=\bfseries, breakable}
\newtcolorbox{intuitionbox}[1][]{colback=yellow!5!white, colframe=yellow!60!black, title=#1, fonttitle=\bfseries, breakable}
\newtcolorbox{critiquebox}[1][]{colback=red!3!white, colframe=red!40!black, title=#1, fonttitle=\bfseries, breakable, leftrule=3pt, rightrule=0.5pt, toprule=0.5pt, bottomrule=0.5pt}

% 代码样式
\lstdefinestyle{python}{
    language=Python,
    basicstyle=\small\ttfamily,
    keywordstyle=\color{blue!70!black}\bfseries,
    commentstyle=\color{codegreen}\itshape,
    stringstyle=\color{red!60!black},
    backgroundcolor=\color{codebg},
    frame=single,
    framerule=0.4pt,
    rulecolor=\color{gray!50},
    breaklines=true,
    showstringspaces=false,
    numbers=left,
    numberstyle=\tiny\color{gray},
    tabsize=4,
}

\newcommand{\R}{\mathbb{R}}
\newcommand{\dd}{\,\mathrm{d}}
\newcommand{\pp}{\partial}
\newcommand{\Ta}{T_{\mathrm{a}}}
\newcommand{\Tset}{T_{\mathrm{set}}}
\newcommand{\Umax}{U_{\mathrm{max}}}

\pagestyle{fancy}
\fancyhead[L]{\small Group Project --- 实验配置专题}
\fancyhead[R]{\small \today}
\fancyfoot[C]{\thepage}

\title{\textbf{实验配置与控制策略调参指南} \\[6pt]
\large 如何使用 \texttt{Week 4\_me/Code} 调配仿真实验}
\author{Andrea Zhang --- MATH3060/1 Group Project}
\date{2026 年 2 月}

\begin{document}
\maketitle
\tableofcontents
\newpage

%% =======================================================================
\part{基础概念:代码架构与实验流程}
%% =======================================================================

\section{文件夹结构总览}

本实验工具箱已自包含在 \texttt{Week 4\_me/} 文件夹下,不依赖上级目录。

\begin{flowbox}[文件夹结构]
\begin{lstlisting}[style=python, numbers=none]
Week 4_me/
  Code/                      # 自包含代码副本
    models/
      pde_2d_model.py        # 核心: 2D 热方程求解器
      ode_model.py           # ODE 模型 (0D)
      pde_1d_model.py        # 1D PDE 模型
    controllers/
      bang_bang.py            # Bang-Bang 控制 (基线)
      pid.py                 # PID 控制 (连续反馈)
      lqr.py                 # LQR 最优控制
      pontryagin.py          # Pontryagin 开环最优
    utils/
      parameters.py          # 统一物理参数
      metrics.py             # 评价指标 (RMSE, 能耗等)
      plotting.py            # 绘图工具
    tests/                   # 单元测试
    experiments/             # 已有的实验脚本
  scenarios.py               # 6 个场景的配置工厂
  run_scenarios.py           # 统一实验运行脚本
  Scenario_Exploration.ipynb # 教学展示 Notebook
  results/                   # 输出图表
  research_note/             # 本文档所在目录
\end{lstlisting}
\end{flowbox}

\begin{intuitionbox}[设计哲学:模型--控制器--场景 三层解耦]
整个代码的设计遵循\textbf{三层分离}原则:
\begin{enumerate}[leftmargin=*]
    \item \textbf{模型层}(\texttt{models/}):定义物理方程和求解方法,不知道控制器是什么
    \item \textbf{控制层}(\texttt{controllers/}):定义控制策略 $u(t, T)$,不知道空间在几维
    \item \textbf{场景层}(\texttt{scenarios.py}):定义边界条件参数,不知道用哪种控制器
\end{enumerate}
三者通过统一接口 \texttt{u\_func(t, T\_therm) $\to$ float} 连接。更换控制器只需\textbf{改一行代码}。
\end{intuitionbox}

\section{运行一次实验的完整流程}

\begin{keybox}[从零到图的四步流程]
\begin{enumerate}[label=\textbf{Step \arabic*}, leftmargin=3em]
    \item \textbf{选择场景}:调用 \texttt{scenarios.py} 中的工厂函数,获取配置字典 \texttt{cfg}
    \item \textbf{构建模型}:用 \texttt{cfg} 构建 \texttt{HeatEquation2D} 实例
    \item \textbf{选择控制器}:实例化控制器,包装为 \texttt{u\_func(t, T\_therm)}
    \item \textbf{运行仿真}:调用 \texttt{model.simulate(u\_func)},获取 $(t, T_{\text{field}}, T_{\text{therm}})$
\end{enumerate}
\end{keybox}

最小可运行示例(在 \texttt{Week 4\_me/} 目录下):

\begin{lstlisting}[style=python]
import sys; sys.path.insert(0, './Code')
import numpy as np
from models.pde_2d_model import HeatEquation2D
from controllers.bang_bang import BangBangController
from scenarios import make_baseline

# Step 1: 选场景
cfg = make_baseline()

# Step 2: 建模型
model = HeatEquation2D(
    Lx=cfg['Lx'], Ly=cfg['Ly'],
    nx=cfg.get('nx', 51), ny=cfg.get('ny', 41),
    heater_pos=cfg['heater_pos'],
    thermostat_pos=cfg['thermostat_pos'],
    wall_h=cfg.get('wall_h'),
    domain_mask=cfg.get('domain_mask'),
    h_updater=cfg.get('h_updater'),
)

# Step 3: 选控制器
ctrl = BangBangController()
def u_func(t, T_therm):
    u = ctrl.get_u(t, T_therm)
    if ctrl._on and T_therm > ctrl.T_set + ctrl.delta:
        ctrl.switch(t, T_therm)
    elif not ctrl._on and T_therm < ctrl.T_set - ctrl.delta:
        ctrl.switch(t, T_therm)
    return u

# Step 4: 跑仿真
t, T_field, T_therm = model.simulate(u_func, t_end=120)
print(f"Final thermostat: {T_therm[-1]:.1f} C")
\end{lstlisting}

\begin{notebox}[快速运行命令]
也可以直接用命令行:
\begin{lstlisting}[style=python, numbers=none]
# 跑所有 6 个场景 (完整 120 min, 约需 5-10 分钟)
python run_scenarios.py

# 只跑 S1 和 S2
python run_scenarios.py S1 S2

# 快速测试模式 (30 min, 约需 1 分钟)
python run_scenarios.py --quick
\end{lstlisting}
\end{notebox}


%% =======================================================================
\part{场景配置:如何修改物理参数}
%% =======================================================================

\section{场景配置字典的结构}

每个场景由一个 Python 字典(dict)描述,包含以下键:

\begin{comparebox}[场景配置字典的键说明]
\renewcommand{\arraystretch}{1.3}
\begin{tabular}{l l l l}
\toprule
\textbf{键} & \textbf{类型} & \textbf{含义} & \textbf{是否必须} \\
\midrule
\texttt{name} & str & 场景标识名 & 是 \\
\texttt{Lx}, \texttt{Ly} & float & 房间长、宽 (m) & 是 \\
\texttt{nx}, \texttt{ny} & int & $x$, $y$ 方向网格点数 & 否 (默认 51, 41) \\
\texttt{wall\_h} & dict & 四面墙的 $h$ 数组 & 否 (默认全 0.5) \\
\texttt{heater\_pos} & (x, y) & 加热器位置 & 是 \\
\texttt{thermostat\_pos} & (x, y) & 恒温器位置 & 是 \\
\texttt{domain\_mask} & ndarray (bool) & 域 mask (L 形等) & 否 (默认全 True) \\
\texttt{h\_updater} & callable & 时变 BC 回调 & 否 (默认无) \\
\bottomrule
\end{tabular}
\end{comparebox}

\section{修改场景参数的常见操作}

\subsection{操作 1:改变窗户大小和材质}

打开 \texttt{scenarios.py},找到 \texttt{make\_window()} 函数:

\begin{lstlisting}[style=python]
def make_window():
    h_south = np.full(NX, H_WALL)  # 先全填 0.5
    x = np.linspace(0, 5, NX)
    # 在 x in [1.5, 3.5] 设为窗户 h=2.5
    h_south[(x >= 1.5) & (x <= 3.5)] = 2.5
    # ^^^^^^^^  改这两个数字调窗户位置
    # ^^^^^^^^^^^^^^^^^^^^^^^^^^^^  改这个值调窗户材质
    ...
\end{lstlisting}

\begin{notebox}[常用窗户 $h$ 值参考]
\renewcommand{\arraystretch}{1.2}
\begin{tabular}{l c l}
\toprule
\textbf{窗户类型} & \textbf{$h$ (m$^{-1}$)} & \textbf{来源} \\
\midrule
单层玻璃 & 2.5 & \cite{mepacademy2024walls} \\
双层中空玻璃 & 1.0 & \cite{incropera2007heat} \\
三层 Low-E 玻璃 & 0.6 & 估计值 \\
普通外墙 (基准) & 0.5 & 项目设定 \\
\bottomrule
\end{tabular}
\end{notebox}

\subsection{操作 2:改变房间形状}

房间形状由 \texttt{Lx}, \texttt{Ly} 和 \texttt{domain\_mask} 共同决定。

\begin{lstlisting}[style=python]
# 例: 创建一个 10x3 米的走廊
def make_corridor():
    nx, ny = 101, 31
    wall_h = {
        'south': np.full(nx, 0.5),  # 外墙
        'north': np.full(nx, 0.5),  # 外墙 (两面朝外)
        'west':  np.full(ny, 0.0),
        'east':  np.full(ny, 0.0),
    }
    return dict(name="Corridor_10x3", Lx=10, Ly=3,
                nx=nx, ny=ny, wall_h=wall_h,
                heater_pos=(5.0, 1.5),
                thermostat_pos=(5.0, 1.5))
\end{lstlisting}

\begin{warnbox}[网格密度一致性]
改变房间尺寸时,务必同步调整 \texttt{nx}, \texttt{ny} 以保持网格密度 $\Delta x \approx 0.1$ m:
\begin{equation}
    n_x = \frac{L_x}{0.1} + 1, \qquad n_y = \frac{L_y}{0.1} + 1
\end{equation}
否则可能因网格太粗导致数值不稳定,或因网格太密导致运行时间过长。
\end{warnbox}

\subsection{操作 3:调整加热器和恒温器位置}

\begin{lstlisting}[style=python]
# 在 make_baseline() 或任何场景工厂中修改:
heater_pos=(1.0, 0.5),     # 加热器: (x, y) 坐标
thermostat_pos=(4.0, 4.0),  # 恒温器: (x, y) 坐标
\end{lstlisting}

\begin{intuitionbox}[加热器位置的物理含义]
\texttt{heater\_pos} 指定了一个 2D Gaussian 热源的中心位置。热源的空间分布为
\begin{equation}
    S(x,y) = \frac{u(t) \cdot G(x,y)}{L_x L_y}, \qquad G(x,y) = \exp\!\left(-\frac{(x-x_h)^2 + (y-y_h)^2}{2\sigma^2}\right)
\end{equation}
其中 $\sigma = 0.5$ m(默认半径)。加热器``辐射''范围约 $2\sigma = 1$ m。

\textbf{位置选择原则}:
\begin{itemize}[leftmargin=*]
    \item 加热器靠近外墙 $\to$ 直接对抗热损失,但加热器附近过热
    \item 加热器在房间中心 $\to$ 温度更均匀,但外墙附近偏冷
    \item 恒温器应代表``人体感知温度'',通常放在房间中部
\end{itemize}
\end{intuitionbox}

\subsection{操作 4:创建新的不规则房间}

\begin{lstlisting}[style=python]
# 例: T 形房间
def make_T_shape():
    nx, ny = 51, 51
    x = np.linspace(0, 5, nx)
    y = np.linspace(0, 5, ny)
    X, Y = np.meshgrid(x, y, indexing='ij')

    mask = np.zeros((nx, ny), dtype=bool)
    # 下部横条: x in [0,5], y in [0,2]
    mask[(Y >= 0) & (Y <= 2)] = True
    # 上部竖条: x in [1.5,3.5], y in [2,5]
    mask[(X >= 1.5) & (X <= 3.5) & (Y >= 2)] = True

    wall_h = {
        'south': np.full(nx, 0.5),
        'north': np.full(nx, 0.0),
        'west':  np.full(ny, 0.0),
        'east':  np.full(ny, 0.0),
    }
    return dict(name="T_Shape", Lx=5, Ly=5, nx=nx, ny=ny,
                wall_h=wall_h, domain_mask=mask,
                heater_pos=(2.5, 1.0),
                thermostat_pos=(2.5, 1.0))
\end{lstlisting}

\subsection{操作 5:添加时变边界条件}

\begin{lstlisting}[style=python]
# 例: 东墙的通风口, 在 t=50~70 min 打开
def make_east_vent(t_open=50, t_close=70):
    def h_updater(t, model):
        h_east = np.full(model.ny, 0.0)
        if t_open <= t <= t_close:
            y = np.linspace(0, model.Ly, model.ny)
            h_east[(y >= 1.0) & (y <= 3.0)] = 5.0
        model.h_east = h_east
    ...
    return dict(..., h_updater=h_updater)
\end{lstlisting}

\begin{warnbox}[\texttt{h\_updater} 的注意事项]
\begin{enumerate}[leftmargin=*]
    \item Callback 在\textbf{每次 RHS 求值时调用}(一次仿真可能调用数万次),应尽量轻量
    \item 修改的是 \texttt{model.h\_east} 等属性(直接赋值),不需要返回值
    \item $h$ 在 $t_{\text{open}}$ 处的突变会让求解器自动缩小步长,这是正确行为
\end{enumerate}
\end{warnbox}


%% =======================================================================
\part{控制策略切换与调参}
%% =======================================================================

\section{四种控制器的数学定义}

\begin{comparebox}[四种控制策略一览]
\renewcommand{\arraystretch}{1.5}
\begin{tabular}{l l l l}
\toprule
\textbf{策略} & \textbf{控制律 $u(t, T)$} & \textbf{关键参数} & \textbf{定位} \\
\midrule
Bang-Bang & $u = \begin{cases} \Umax & T < \Tset - \delta \\ 0 & T > \Tset + \delta \end{cases}$ & $\delta$ (滞回带宽) & 基线 \\[10pt]
PID & $u = K_p e + K_i \int e\dd t + K_d \dot{e}$ & $K_p, K_i, K_d$ & 工程标准 \\[10pt]
LQR & $u = -K(T - \Tset) + u_{\text{ss}}$ & $Q, R$ (代价权重) & 数学最优 \\[10pt]
Pontryagin & $u^*(t) = \text{clip}\!\left(-\frac{\lambda(t)}{2R}, 0, \Umax\right)$ & $Q, R, t_f$ & 理论极限 \\
\bottomrule
\end{tabular}
\end{comparebox}

\begin{keybox}[统一接口:\texttt{get\_u(t, T)} $\to$ float]
所有控制器都实现了相同的方法签名:
\begin{lstlisting}[style=python, numbers=none]
class SomeController:
    def get_u(self, t, T) -> float:
        """输入: 当前时间 t, 恒温器温度 T
           输出: 加热功率 u in [0, U_max]"""
\end{lstlisting}
因此切换控制器\textbf{只需改一行代码},模型层和场景层完全不用动。
\end{keybox}

\section{如何切换控制器}

\subsection{在 \texttt{run\_scenarios.py} 中切换}

打开 \texttt{run\_scenarios.py},找到 \texttt{run\_one()} 函数中的控制器实例化:

\begin{lstlisting}[style=python]
def run_one(cfg, t_end=T_END):
    model = build_model(cfg)
    # ===== 在这里切换控制器 =====
    ctrl = BangBangController()          # 当前: Bang-Bang
    # ctrl = PIDController(Kp=2.0, Ki=0.1, Kd=0.5, dt=0.5)  # PID
    # ctrl = LQRController(Q=1.0, R=0.01)                    # LQR
    ...
\end{lstlisting}

\begin{warnbox}[Bang-Bang 需要额外的滞回逻辑]
Bang-Bang 控制器使用事件检测的滞回机制,在 \texttt{run\_one()} 中 \texttt{u\_func} 包含了手动的 switch 逻辑。如果切换到 PID 或 LQR,\texttt{u\_func} 可以简化为:
\begin{lstlisting}[style=python, numbers=none]
def u_func(t, T_therm):
    return ctrl.get_u(t, T_therm)  # PID/LQR 无需手动 switch
\end{lstlisting}
\end{warnbox}

\subsection{在 Notebook 中切换}

\texttt{Scenario\_Exploration.ipynb} 中的 \texttt{run\_scenario()} 函数与 \texttt{run\_one()} 逻辑相同,改同一处即可。

\subsection{同时对比多个控制器}

\begin{lstlisting}[style=python]
# 对同一个场景用不同控制器跑, 然后对比
from controllers.bang_bang import BangBangController
from controllers.pid import PIDController
from controllers.lqr import LQRController

cfg = make_baseline()
controllers = {
    'Bang-Bang': BangBangController(),
    'PID':       PIDController(Kp=2.0, Ki=0.1, Kd=0.5, dt=0.5),
    'LQR':       LQRController(Q=1.0, R=0.01),
}

for name, ctrl in controllers.items():
    model = build_model(cfg)
    def u_func(t, T, _ctrl=ctrl):  # 闭包捕获 ctrl
        return _ctrl.get_u(t, T)
    t, T_field, T_therm = model.simulate(u_func, t_end=120)
    print(f"{name}: Final T_therm = {T_therm[-1]:.1f} C")
\end{lstlisting}

\section{Bang-Bang 控制器调参}

\subsection{数学模型}

带滞回的 Bang-Bang 控制器是一个\textbf{混合动力系统} \cite{goebel2009hybrid}:

\begin{equation}
    u(t) = \begin{cases}
        \Umax & \text{if heater is ON} \\
        0 & \text{if heater is OFF}
    \end{cases}
\end{equation}

切换条件:
\begin{align}
    \text{ON} \to \text{OFF}: \quad & T > \Tset + \delta \\
    \text{OFF} \to \text{ON}: \quad & T < \Tset - \delta
\end{align}

\subsection{可调参数}

\begin{comparebox}[Bang-Bang 参数调整指南]
\renewcommand{\arraystretch}{1.3}
\begin{tabular}{l c l l}
\toprule
\textbf{参数} & \textbf{默认值} & \textbf{调大的效果} & \textbf{调小的效果} \\
\midrule
\texttt{delta} ($\delta$) & 0.5\textdegree C & 切换少,温度波动大 & 切换频繁,接近 Zeno \\
\texttt{U\_max} ($\Umax$) & 15 & 升温快,超调大 & 升温慢,可能达不到设定 \\
\texttt{T\_set} ($\Tset$) & 20\textdegree C & 更高目标温度 & 更低目标温度 \\
\bottomrule
\end{tabular}
\end{comparebox}

\begin{lstlisting}[style=python]
# 调参示例
ctrl = BangBangController(
    T_set=22.0,   # 提高设定温度
    U_max=15,     # 保持默认功率
    delta=1.0,    # 加宽滞回带 -> 减少开关次数
)
\end{lstlisting}

\begin{intuitionbox}[滞回带宽 $\delta$ 的权衡]
$\delta$ 控制了\textbf{舒适度 vs 设备磨损}的权衡:
\begin{itemize}[leftmargin=*]
    \item $\delta = 0$: 温度精确跟踪 $\Tset$,但切换无穷频繁(\textbf{Zeno 效应},见 \cite{zhang2001zeno})
    \item $\delta = 0.5$: 温度在 $[\Tset - 0.5, \Tset + 0.5]$ 范围内波动,切换次数适中
    \item $\delta = 2.0$: 温度在 $[18, 22]$\textdegree C 范围内大幅波动,切换很少
\end{itemize}
实际恒温器通常取 $\delta \approx 0.5$--$1.0$\textdegree C。
\end{intuitionbox}


\section{PID 控制器调参}

\subsection{数学模型}

离散 PID 控制律 \cite{astrom2021feedback}:
\begin{equation}
    u(t) = \underbrace{K_p \cdot e(t)}_{\text{比例}} + \underbrace{K_i \int_0^t e(\tau)\dd\tau}_{\text{积分}} + \underbrace{K_d \frac{\dd e}{\dd t}}_{\text{微分}}, \quad e(t) = \Tset - T(t)
\end{equation}
输出限幅至 $[0, \Umax]$,带 \textbf{anti-windup}(积分饱和时停止积分累积)。

\subsection{三个增益参数的物理含义}

\begin{comparebox}[PID 三参数的作用]
\renewcommand{\arraystretch}{1.3}
\begin{tabular}{c l l l}
\toprule
\textbf{参数} & \textbf{作用} & \textbf{过大} & \textbf{过小} \\
\midrule
$K_p$ & 放大当前误差 & 响应快但振荡 & 响应慢、稳态误差大 \\
$K_i$ & 消除稳态误差 & 超调严重、积分饱和 & 仍有残余误差 \\
$K_d$ & 抑制变化率(阻尼) & 对噪声敏感 & 振荡不易消除 \\
\bottomrule
\end{tabular}
\end{comparebox}

\subsection{调参方法}

\begin{keybox}[PID 调参三步法]
\begin{enumerate}[label=\textbf{\arabic*.}, leftmargin=2em]
    \item \textbf{先调 $K_p$}:令 $K_i = K_d = 0$,逐渐增大 $K_p$ 直到系统快速响应但不振荡
    \item \textbf{再加 $K_i$}:逐渐增大 $K_i$,消除稳态误差,但不要太大以免超调
    \item \textbf{最后加 $K_d$}:增大 $K_d$ 以抑制超调和振荡
\end{enumerate}
\end{keybox}

\begin{lstlisting}[style=python]
# 方法 1: 手动调参
ctrl = PIDController(Kp=3.0, Ki=0.05, Kd=1.0, dt=0.5)

# 方法 2: Ziegler-Nichols 自动估计
from controllers.pid import ziegler_nichols_tuning
Kp, Ki, Kd = ziegler_nichols_tuning(k_cool=0.1)
print(f"Z-N: Kp={Kp:.2f}, Ki={Ki:.2f}, Kd={Kd:.2f}")
ctrl = PIDController(Kp=Kp, Ki=Ki, Kd=Kd, dt=0.5)
\end{lstlisting}

\begin{notebox}[Ziegler-Nichols 调参的原理]
代码中的 \texttt{ziegler\_nichols\_tuning(k\_cool)} 基于房间 ODE 模型 $\dd T/\dd t = -k(T - \Ta) + u$ 的时间常数 $\tau = 1/k$:
\begin{align}
    K_p &= 0.6 \cdot \Umax / 15 \approx 0.6 \\
    K_i &= K_p / (0.5\tau) = K_p \cdot 2k \\
    K_d &= K_p \cdot 0.125\tau
\end{align}
这只是起始估计值,通常还需要根据 2D PDE 模型的响应做微调。
\end{notebox}

\begin{warnbox}[PID 用于 2D PDE 时的注意事项]
PID 是为 ODE(0D 模型)设计的。在 2D PDE 中,恒温器只读取\textbf{一个点}的温度,但控制器的输出影响整个温度场。这意味着:
\begin{itemize}[leftmargin=*]
    \item 恒温器位置对 PID 性能影响巨大(比 Bang-Bang 更敏感)
    \item $K_d$ 的微分作用可能被 2D 扩散效应``稀释'',需要适当增大
    \item \texttt{dt} 参数应与仿真的时间步一致(默认 0.5 min = \texttt{t\_eval} 的间隔)
\end{itemize}
\end{warnbox}


\section{LQR 控制器调参}

\subsection{数学模型}

LQR 最小化二次代价泛函 \cite{anderson1990lqr}:
\begin{equation}
    J = \int_0^\infty \left[ Q \cdot (T - \Tset)^2 + R \cdot u^2 \right] \dd t
\end{equation}

最优反馈律:
\begin{equation}
    \boxed{u^* = -K \cdot (T - \Tset) + u_{\text{ss}}}
\end{equation}
其中 $K$ 由 Riccati 方程求得,$u_{\text{ss}} = k(\Tset - \Ta)$ 是稳态前馈项。

\subsection{两个权重参数的物理含义}

\begin{intuitionbox}[$Q$ 和 $R$ 的直觉理解]
\begin{itemize}[leftmargin=*]
    \item $Q$ = \textbf{``温度偏差有多贵''}:$Q$ 越大,控制器越不能容忍偏离设定值
    \item $R$ = \textbf{``用电有多贵''}:$R$ 越大,控制器越``省电''但允许温度偏差
    \item \textbf{核心权衡}:$Q/R$ 比值决定了``舒适 vs 节能''的平衡点
\end{itemize}
\end{intuitionbox}

\begin{comparebox}[LQR $Q/R$ 参数效果]
\renewcommand{\arraystretch}{1.3}
\begin{tabular}{c c l l}
\toprule
$Q$ & $R$ & \textbf{行为} & \textbf{适用场景} \\
\midrule
1.0 & 0.01 & 激进追踪,高能耗 & 手术室、精密实验室 \\
1.0 & 0.1 & 平衡模式 & 普通办公室 \\
1.0 & 1.0 & 节能模式,温度波动大 & 仓库、无人区域 \\
\bottomrule
\end{tabular}
\end{comparebox}

\begin{lstlisting}[style=python]
# 基础用法
ctrl = LQRController(Q=1.0, R=0.01)  # 默认: 激进追踪

# Pareto 参数扫描: 探索 Q/R 空间
from controllers.lqr import pareto_scan
Q_vals = [0.1, 1.0, 10.0]
R_vals = [0.001, 0.01, 0.1, 1.0]
ctrls, params = pareto_scan(Q_vals, R_vals)
# ctrls[i] 对应 params[i] = (Q, R)
\end{lstlisting}

\begin{derivbox}[Riccati 方程的标量形式]
对于一阶 ODE 模型 $\dot{x} = -kx + u$($x = T - \Tset$),代数 Riccati 方程为
\begin{equation}
    -kP + P(-k) - \frac{P^2}{R} + Q = 0
\end{equation}
解为
\begin{equation}
    P = R\left(-k + \sqrt{k^2 + Q/R}\right), \qquad K = P/R = -k + \sqrt{k^2 + Q/R}
\end{equation}
当 $Q/R \gg k^2$ 时,$K \approx \sqrt{Q/R}$(高增益反馈)。

\textbf{注意}:这个 $K$ 是基于 ODE 模型计算的。将其用于 2D PDE 时,控制器只看恒温器一点的温度,与 ODE 中``看整间房温度''不同,因此效果会有差异。
\end{derivbox}


\section{Pontryagin 最优控制(开环)}

\subsection{与其他控制器的本质区别}

\begin{keybox}[Pontryagin: 开环 vs 闭环]
Pontryagin 控制器是\textbf{预计算的开环控制}:
\begin{itemize}[leftmargin=*]
    \item 求解两点边值问题(TPBVP),得到整条最优轨迹 $u^*(t)$
    \item 运行时直接\textbf{插值查表},不读取当前温度
    \item 对模型扰动(开门、窗户变化)\textbf{没有反馈能力}
\end{itemize}
因此 Pontryagin 适合作为``理论最优基准'',不适合实际控制。
\end{keybox}

\begin{lstlisting}[style=python]
from controllers.pontryagin import PontryaginController

# 默认参数 (基于 ODE 模型)
ctrl = PontryaginController(Q=1.0, R=0.01, t_end=120)
# 内部自动求解 TPBVP, 可能需要几秒

# 使用方式与其他控制器完全相同
u = ctrl.get_u(t=30, T=18.5)  # 但实际上忽略了 T, 直接插值
\end{lstlisting}


%% =======================================================================
\part{评价指标与结果分析}
%% =======================================================================

\section{六项评价指标}

所有控制策略用统一指标对比(定义在 \texttt{Code/utils/metrics.py}):

\begin{comparebox}[评价指标体系]
\renewcommand{\arraystretch}{1.5}
\begin{tabular}{l l l}
\toprule
\textbf{指标} & \textbf{公式} & \textbf{含义} \\
\midrule
能耗 $E$ & $\displaystyle\int_0^{t_f} u(t)\dd t$ & 总加热能量 \\[6pt]
温度偏差 RMSE & $\displaystyle\sqrt{\frac{1}{t_f}\int_0^{t_f}(T - \Tset)^2 \dd t}$ & 平均偏离程度 \\[6pt]
最大超调 & $\max(T(t) - \Tset, 0)$ & 峰值过热 \\[6pt]
稳定时间 $t_s$ & 首次进入 $\Tset \pm 0.5$\textdegree C 并保持 & 响应速度 \\[6pt]
切换次数 $N_s$ & 加热器 ON$\leftrightarrow$OFF 的次数 & 设备磨损 \\[6pt]
统一代价 $J$ & $\displaystyle\int_0^{t_f}\!\left[Q(T-\Tset)^2 + Ru^2\right]\dd t$ & 综合权衡 \\
\bottomrule
\end{tabular}
\end{comparebox}

\begin{lstlisting}[style=python]
from utils.metrics import compute_all_metrics
from utils.parameters import T_SET

metrics = compute_all_metrics(t, T_therm, u_arr, T_SET)
print(f"Energy:     {metrics['energy']:.1f}")
print(f"RMSE:       {metrics['rmse']:.2f} C")
print(f"Overshoot:  {metrics['max_overshoot']:.2f} C")
print(f"Settle:     {metrics['settling_time']:.1f} min")
print(f"Switches:   {metrics['switching_count']}")
print(f"Cost J:     {metrics['unified_cost']:.1f}")
\end{lstlisting}

\section{空间不均匀度(PDE 特有)}

对于 2D PDE 模型,温度场的空间分布不均匀度也是重要指标:

\begin{equation}
    \sigma_T = \text{std}\left(T(x,y,t_f)\right) = \sqrt{\frac{1}{|\Omega|}\int_\Omega (T - \bar{T})^2 \dd\Omega}
\end{equation}

\begin{lstlisting}[style=python]
# 终态温度场的空间不均匀度
T_final = T_field[:, :, -1]
mask = cfg.get('domain_mask')
if mask is not None:
    T_active = T_final[mask]
else:
    T_active = T_final.ravel()
nonuniformity = np.std(T_active)
print(f"Spatial std: {nonuniformity:.2f} C")
\end{lstlisting}


%% =======================================================================
\section{批判性分析}
%% =======================================================================

\begin{critiquebox}[控制策略选择的批判性评价]
\textbf{优势}:
\begin{itemize}[leftmargin=*]
    \item 统一接口设计使策略切换极为简单,有利于系统化对比
    \item 四种控制器覆盖了从最简单(Bang-Bang)到最理论(Pontryagin)的完整谱系
    \item 评价指标全面,涵盖了能耗、舒适度、设备寿命三个维度
\end{itemize}

\textbf{局限}:
\begin{itemize}[leftmargin=*]
    \item PID 和 LQR 的参数是基于 ODE(0D)模型设计的,直接用于 2D PDE 可能不是最优
    \item 所有控制器只读取\textbf{一个点}(恒温器)的温度,无法感知空间分布
    \item Pontryagin 是开环控制,对扰动(开门、日照变化)无反馈能力
    \item 未考虑执行器动态(加热器不能瞬间开关,有启动延迟)
\end{itemize}

\textbf{未解决的问题}:
\begin{itemize}[leftmargin=*]
    \item 如何为 2D PDE 模型专门设计控制器?(需要分布参数控制理论)
    \item 多恒温器、多加热器的协调控制问题
    \item 模型预测控制(MPC)是否能更好地处理时变扰动?
\end{itemize}

\textbf{对本项目的启示}:当前四种策略足以展示``不同数学工具的应用'',但报告中应明确指出 0D 控制器用于 2D 模型的局限性,并将其作为 Discussion 部分的亮点。
\end{critiquebox}


%% =======================================================================
\section{快速查阅卡片}
%% =======================================================================

\begin{keybox}[常用命令速查]
\renewcommand{\arraystretch}{1.4}
\begin{tabular}{l l}
\toprule
\textbf{操作} & \textbf{代码} \\
\midrule
运行全部场景 & \texttt{python run\_scenarios.py} \\
快速测试 & \texttt{python run\_scenarios.py --quick} \\
只跑 S1+S4 & \texttt{python run\_scenarios.py S1 S4} \\
切到 PID & \texttt{ctrl = PIDController(Kp=2, Ki=0.1, Kd=0.5, dt=0.5)} \\
切到 LQR & \texttt{ctrl = LQRController(Q=1.0, R=0.01)} \\
Z-N 自动调参 & \texttt{Kp,Ki,Kd = ziegler\_nichols\_tuning(0.1)} \\
计算指标 & \texttt{m = compute\_all\_metrics(t, T, u, T\_SET)} \\
跑测试 & \texttt{cd Code \&\& python -m pytest tests/ -v} \\
\bottomrule
\end{tabular}
\end{keybox}


%% =======================================================================
\section{自检清单}
%% =======================================================================

\begin{enumerate}[label=$\square$, leftmargin=2em]
    \item 能不看文档,从零运行一个场景并获得温度场图
    \item 能修改 \texttt{scenarios.py} 创建一个新场景(如 T 形房间)
    \item 能在 \texttt{run\_scenarios.py} 中把 Bang-Bang 切换为 PID 并跑通
    \item 能解释 PID 三个参数 $K_p, K_i, K_d$ 分别控制什么
    \item 能解释 LQR 中 $Q/R$ 比值对控制行为的影响
    \item 能用 \texttt{compute\_all\_metrics} 计算出全部 6 项指标
    \item 能说出 Pontryagin 与其他三种控制器的本质区别(开环 vs 闭环)
    \item 能向老师解释代码的三层解耦设计
\end{enumerate}

\printbibliography

\end{document}
