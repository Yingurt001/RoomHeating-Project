\documentclass[11pt,a4paper]{article}
\usepackage[utf8]{inputenc}
\usepackage{amsmath,amssymb}
\usepackage{graphicx}
\usepackage{booktabs}
\usepackage{hyperref}
\usepackage[margin=2.5cm]{geometry}
\usepackage{float}

\title{Week 2 Report: 1D Heat Equation with Bang-Bang Control\\
\large Thermostat Placement Experiment}
\author{Group Project - Room Heating System}
\date{\today}

\begin{document}
\maketitle

\section{Introduction}

This report documents the numerical implementation and validation of the 1D heat equation with Bang-Bang thermostat control. The key objectives are:
\begin{enumerate}
    \item Validate numerical solution against analytical steady-state solution
    \item Study the effect of thermostat placement on control performance
    \item Evaluate results against thermal comfort standards (ASHRAE 55)
\end{enumerate}

\section{Mathematical Model}

\subsection{Governing Equation}

The 1D heat conduction equation with a point heat source:
\begin{equation}
    \frac{\partial T}{\partial t} = \alpha \frac{\partial^2 T}{\partial x^2} + \frac{\alpha}{k_1} P \cdot \delta(x - x_0)
\end{equation}
where:
\begin{itemize}
    \item $T(x,t)$: temperature field
    \item $\alpha$: thermal diffusivity (m$^2$/min)
    \item $k_1$: thermal conductivity
    \item $P$: heater power
    \item $\delta(x-x_0)$: Dirac delta function (point heat source at $x_0$)
\end{itemize}

\textbf{Key Correction}: The heat source term must include the factor $\alpha/k_1$ to ensure the numerical solution converges to the correct analytical steady-state.

\subsection{Boundary Conditions}

\begin{itemize}
    \item \textbf{Left (x=0)}: Neumann (insulated wall): $\displaystyle\frac{\partial T}{\partial x}\bigg|_{x=0} = 0$
    \item \textbf{Right (x=L)}: Robin (convective wall): $\displaystyle -k_1 \frac{\partial T}{\partial x}\bigg|_{x=L} = h(T(L,t) - T_a)$
\end{itemize}

\subsection{Analytical Steady-State Solution}

At steady state ($\partial T/\partial t = 0$):
\begin{equation}
    T(x) = \begin{cases}
        T_a + \frac{P}{h} + \frac{P}{k_1}(L - x_0), & 0 \leq x \leq x_0 \\[8pt]
        T_a + \frac{P}{h} + \frac{P}{k_1}(L - x), & x_0 < x \leq L
    \end{cases}
\end{equation}

\section{Thermal Comfort Standards}

According to ASHRAE Standard 55 and ISO 7730:

\begin{table}[H]
\centering
\begin{tabular}{lcc}
\toprule
Season & Temperature Range & Notes \\
\midrule
Winter & 20--24$^\circ$C & Heating season \\
Summer & 22.5--26$^\circ$C & Cooling season \\
Optimal & 18--26$^\circ$C & Year-round comfort \\
\bottomrule
\end{tabular}
\caption{Thermal comfort temperature ranges}
\end{table}

Additional requirements:
\begin{itemize}
    \item Vertical temperature difference (head to ankle): $\leq 3^\circ$C (seated)
    \item Air velocity: $< 0.15$ m/s to avoid draft sensation
\end{itemize}

\section{Numerical Implementation}

\subsection{Parameters}

\begin{table}[H]
\centering
\begin{tabular}{lcc}
\toprule
Parameter & Symbol & Value \\
\midrule
Room length & $L$ & 5.0 m \\
Thermal diffusivity & $\alpha$ & 0.5 m$^2$/min \\
Thermal conductivity & $k_1$ & 1.0 \\
Convection coefficient & $h$ & 0.5 \\
Maximum heating power & $U_{\max}$ & 5.0 \\
Outdoor temperature & $T_a$ & 5$^\circ$C \\
Setpoint temperature & $T_{\text{set}}$ & 20$^\circ$C \\
Hysteresis width & $\delta$ & 0.5$^\circ$C \\
\bottomrule
\end{tabular}
\caption{Simulation parameters}
\end{table}

\subsection{Time Scale Analysis}

The diffusion time scale is:
\begin{equation}
    \tau = \frac{L^2}{\alpha} = \frac{25}{0.5} = 50 \text{ min}
\end{equation}

The system reaches approximate steady state after $4$--$5\tau \approx 200$--$250$ min.

\section{Results and Figure Explanations}

\subsection{Figure 1: Numerical Validation}

\begin{figure}[H]
\centering
\includegraphics[width=0.95\textwidth]{Result_CN/fig1_validation.png}
\caption{Validation of numerical solution against analytical steady-state}
\end{figure}

\textbf{Left panel}: Temperature evolution from initial condition ($T_a = 5^\circ$C) to steady state. Color gradient shows time progression from $t=0$ (dark) to $t=3\tau$ (light). The red dashed line is the analytical steady-state solution.

\textbf{Right panel}: Comparison at final time. Maximum error = 0.49$^\circ$C, confirming the numerical implementation is correct.

\textbf{Physical interpretation}: Heat flows from the point source ($x_0 = 2.5$m) toward the convective wall ($x=L$). The left side (insulated) maintains constant temperature, while the right side shows linear decrease due to heat loss through convection.

\subsection{Figure 2: Bang-Bang Control Simulation}

\begin{figure}[H]
\centering
\includegraphics[width=0.95\textwidth]{Result_CN/fig2_simulation.png}
\caption{1D simulation with Bang-Bang control (heater and sensor at $x=2.5$m)}
\end{figure}

\textbf{Top panel (heatmap)}: Spatiotemporal temperature field $T(x,t)$. Horizontal axis is time, vertical axis is position. The cyan dashed line marks the heater/sensor location. Hot colors indicate higher temperatures near the heater.

\textbf{Middle panel}: Temperature time series. Green line = sensor reading; Blue line = room average. The sensor oscillates around the setpoint (20$^\circ$C) within the hysteresis band ($T_L = 19.5^\circ$C to $T_H = 20.5^\circ$C). The green shaded region indicates the ASHRAE comfort zone.

\textbf{Bottom panel}: Heater ON/OFF status. The Bang-Bang controller switches frequently (58 times in 60 min) when the sensor is at the heat source location.

\textbf{Key observation}: While the sensor maintains the setpoint, the room average temperature ($\approx 18^\circ$C) is below the setpoint, and the boundary regions are even colder.

\subsection{Figure 3: Thermostat Placement Comparison}

\begin{figure}[H]
\centering
\includegraphics[width=0.95\textwidth]{Result_CN/fig3_placement.png}
\caption{Effect of thermostat placement on control performance}
\end{figure}

This figure compares five sensor positions ($x_s = 0, 1.25, 2.5, 3.75, 5.0$ m) with the heater fixed at $x_0 = 2.5$ m.

\textbf{Top-left (Sensor Reading)}:
\begin{itemize}
    \item $x_s = 2.5$m (at heater): Sensor oscillates tightly around setpoint
    \item $x_s = 0$m (insulated wall): Slower response, good control
    \item $x_s = 5.0$m (convective wall): Sensor never reaches setpoint; heater runs continuously
\end{itemize}

\textbf{Top-right (Room Average)}:
All configurations achieve similar room average temperature ($\approx 17$--$20^\circ$C), but with different dynamics.

\textbf{Bottom-left (Coldest Point)}:
The minimum temperature in the room shows when comfort violations occur. Most configurations have minimum temperatures below the comfort threshold (18$^\circ$C), indicating the room is not uniformly comfortable.

\textbf{Bottom-right (Trade-off Chart)}:
\begin{itemize}
    \item \textbf{Energy}: Sensor at cold locations ($x_s = 3.75$m, $5.0$m) causes continuous heating $\rightarrow$ highest energy use
    \item \textbf{RMSE}: Sensor at the heater ($x_s = 2.5$m) has lowest sensor RMSE but similar room-average RMSE
\end{itemize}

\section{Key Findings}

\begin{enumerate}
    \item \textbf{Numerical validation successful}: Max error $< 0.5^\circ$C after correcting the heat source term.

    \item \textbf{Heat source term correction}: The correct form is $S = (\alpha/k_1) \cdot P \cdot \delta(x-x_0)$, not $P \cdot \delta(x-x_0)$.

    \item \textbf{Thermostat placement matters}:
    \begin{itemize}
        \item At heat source: Tight control but high switching frequency (58 switches)
        \item At cold wall: Heater runs continuously, wastes energy
        \item Optimal: Between heat source and boundary (e.g., $x_s = 1.25$m)
    \end{itemize}

    \item \textbf{Comfort zone challenges}: With a point heat source, achieving uniform room temperature within the comfort zone is difficult. The coldest point typically falls below the 18$^\circ$C comfort minimum.
\end{enumerate}

\section{Recommendations for Future Work}

\begin{enumerate}
    \item Consider distributed heat sources instead of point sources
    \item Implement PID control to reduce switching frequency
    \item Add spatial averaging to the sensor feedback
    \item Study 2D models for more realistic room geometry
\end{enumerate}

\section{References}

\begin{enumerate}
    \item ASHRAE Standard 55-2023: Thermal Environmental Conditions for Human Occupancy
    \item ISO 7730:2005: Ergonomics of the thermal environment
    \item Handnote derivations (Project 2 documentation)
\end{enumerate}

\end{document}
